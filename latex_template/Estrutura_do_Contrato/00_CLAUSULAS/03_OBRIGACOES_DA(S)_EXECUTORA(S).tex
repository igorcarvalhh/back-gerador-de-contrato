\causula{CLÁUSULA TERCEIRA - OBRIGAÇÕES DA(S) \textbf{EXECUTORA(S)}}

A(s) \textbf{EXECUTORA(S)} se comprometem a, sem prejuízo de outras obrigações previstas neste \textbf{CONTRATO}:

## Seguir e obedecer rigorosamente às condições estabelecidas neste \textbf{CONTRATO} e nos demais documentos que o integram;

## Executar, fiel e integralmente, o objeto descrito neste instrumento e no PLANO DE TRABALHO anexo ao presente instrumento, em estrito cumprimento aos prazos definidos, bem como ao Cronograma Físico e Composição da Equipe de Pesquisa e Desenvolvimento estabelecidos nos itens “Recursos Financeiros – Etapas” do PLANO DE TRABALHO anexo;

\xx Recolher os encargos pertinentes, impostos, taxas e arcar com todas as obrigações trabalhistas, contribuições previdenciárias e sociais, seguros de acidente do trabalho e mais todo e qualquer dispêndio da execução do \textbf{PROJETO}, que serão de sua responsabilidade exclusiva, ainda que lançados indevidamente contra a \textbf{COOPERADA(S)} e/ou a \textbf{PROPONENTE};

\xx A eventual ocorrência de processos judiciais ou administrativos propostos contra a \textbf{COOPERADA(S)} e da \textbf{PROPONENTE} de denunciar a(s) \textbf{EXECUTORA(S)} à lide e de seu direito de regresso contra a(s) \textbf{EXECUTORA(S)}:

\xxx  Em qualquer hipótese, ressarcir a \textbf{COOPERADA(S)} e/ou a \textbf{PROPONENTE}, a seu exclusivo critério, reter tais valores nos repasses financeiros devidos à(s) \textbf{EXECUTORA(S)}, até que esta assuma sua integral responsabilidade; e

\xxx  Somado ao acima disposto, em caso de reclamação trabalhista intentada por seu empregado contra a \textbf{COOPERADA(S)} e/ou a \textbf{PROPONENTE}, comparecer espontaneamente em juízo, reconhecendo sua real condição de empregadora e substituindo a(s) \textbf{EXECUTORA(S)} em eventuais processos, até a decisão final.

\xx Comprovar mensalmente, ou sempre que solicitado pela \textbf{COOPERADA(S)} e/ou a \textbf{PROPONENTE}, sob pena de retenção do pagamento das notas fiscais:

\begin{enumerate}[label=\alph*)]
    \item O devido pagamento dos salários e recolhimento de todas as obrigações trabalhistas relativas aos empregados utilizados na execução do objeto deste \textbf{CONTRATO}, sejam próprios ou de suas eventuais \textbf{SUBCONTRATADOS}; e
    \item A sua efetiva regularidade junto aos órgãos do poder público responsáveis pela fiscalização do recolhimento de tributos, encargos e contribuições relacionadas ao objeto deste \textbf{CONTRATO}, tais como, sem, contudo, se limitar a, guias de recolhimento de todos os impostos e contribuições sociais, certidões negativas atualizadas, da Receita Federal, da Procuradoria Geral da Fazenda Nacional (PGFN), da Receita Estadual da Receita Municipal e do INSS.
\end{enumerate}

\xx Manter constantemente atualizados os exames médicos admissionais, periódicos e demissionais dos empregados por si contratados;

\xx Cumprir e fazer respeitar, por todo o seu pessoal, os regulamentos e normas disciplinares, de segurança e higiene existentes no local de trabalho, bem como a fornecer, às suas expensas, refeições, transporte, uniformes e equipamentos de proteção individual aos seus funcionários, os quais deverão sempre portar a devida identificação;

\xx Fornecer todos os Equipamentos de Proteção Individual (EPI) e Coletivo (EPC), necessários à preservação da integridade de seu empregado e terceiros, bem como exigir a sua utilização e responsabilizar-se por sua conservação e reposição, conforme NR-06 da Portaria 3214/MTB/78;

\xx Substituir no prazo de 48 (quarenta e oito) horas contadas da solicitação por escrito da \textbf{COOPERADA(S)} e/ou a \textbf{PROPONENTE}, qualquer membro de sua equipe alocado aos serviços objeto deste \textbf{CONTRATO};

\xx Suprir em tempo hábil qualquer ausência de empregado alocado, de modo a preservar o padrão de qualidade técnica e impedir a descontinuidade na execução do objeto contratado;

\xx Assumir a integral responsabilidade por todos os atos ou omissões seus, de seus \textbf{SUBCONTRATADOS} e dos funcionários, mão de obra, representantes, agentes e pessoal de qualquer natureza fornecidos por si ou por seus \textbf{SUBCONTRATADOS}, obrigando-se a ressarcir quaisquer danos que os mesmos venham a causar à \textbf{COOPERADA(S)} e/ou a \textbf{PROPONENTE} e/ou a terceiros;

% tem link
\xxx  A SUBCONTRATAÇÃO não exime a EXECUTORA das obrigações deste contrato, com destaque à cláusula de prestação de contas citada na CLÁUSULA TERCEIRA, \hyperlink{3.19}{SUBCLÁUSULA “3.19”}.

\xx Fornecer toda infraestrutura e recursos necessários ao atendimento da execução do \textbf{PROJETO}, como laboratórios, linhas telefônicas, computadores, impressoras e despesas de telefonia, hospedagem, alimentação, deslocamento e demais despesas operacionais para execução deste \textbf{PROJETO};

\xx Apresentar, com periodicidade de 4 meses, material gráfico demonstrativo dos resultados parciais e finais do \textbf{PROJETO} para divulgação interna entre os PARTÍCIPES. O material deve ser elaborado pela área de comunicação, marketing ou publicidade da(s) \textbf{EXECUTORA(S)} e deve possuir qualidade condizente a especialidade destas áreas;

\xxx Adicionalmente, deverá ser entregue material gráfico ao final da Fase da CADEIA DE INOVAÇÃO. Este material gráfico deverá conter imagens dos objetos desenvolvidos e deverá sintetizar todas as informações referentes à Fase, também epilogando os materiais anteriormente entregues.

\xx Reparar ou refazer, exclusivamente às suas expensas, suas obrigações que tenham, porventura, sido inadequadamente executados, sanando todos os problemas dentro dos prazos estipulados pela \textbf{COOPERADA(S)} e a \textbf{PROPONENTE};

\xx Redimir atrasos no cumprimento do objeto do \textbf{CONTRATO} a que tiver dado causa, tomando todas as providências no sentido de cumprir o cronograma de execução acertado com a \textbf{COOPERADA(S)} e/ou a \textbf{PROPONENTE} ou por motivo de força maior;

\xx Manter no local onde será executado o objeto deste \textbf{CONTRATO}, pessoa qualificada e credenciada, devidamente indicada por escrito pela \textbf{EXECUTORA(S)}, para supervisionar a execução das atividades e para receber orientações da \textbf{COOPERADA(S)} e/ou a \textbf{PROPONENTE} e repassar às equipes e turmas que as executam;

\xx Realizar a homologação no portal de fornecedores da \textbf{COOPERADA(S)} e da \textbf{PROPONENTE}, observando os seguintes critérios:

\xxx  Criar perfil de homologação referente ao faturamento anual da(s) \textbf{EXECUTORA(S)} junto à \textbf{COOPERADA(S)} e à \textbf{PROPONENTE};

\xxx  O prazo de validade de homologação, qual seja, de (doze) meses, e suas eventuais prorrogações ao longo do \textbf{CONTRATO};

\xxx  Os custos inerentes à homologação e suas renovações são de responsabilidade da(s) \textbf{EXECUTORA(S)};

\xxx  A criação e a manutenção dos dados cadastrais, tais como e sem se limitar a nome empresarial, endereço completo e CNPJ, são de responsabilidade da(s) \textbf{EXECUTORA(S)}, que deverá imputar toda e qualquer informação no Portal de Fornecedores e comunicar prontamente a finalização do cadastro e/ou suas atualizações à \textbf{COOPERADA(S)} e à \textbf{PROPONENTE};

\xx Fornecer à \textbf{COOPERADA(S)} e pela \textbf{PROPONENTE};

\xx Prestar contas de todo e qualquer gasto realizado com os recursos financeiros repassados pelos PARTÍCIPES sob o prisma dos princípios da legalidade, da moralidade, do interesse público, da publicidade e da motivação dos atos, explicitando todas as despesas pagas com recursos destinados à execução do \textbf{PROJETO} através da apresentação de todos os documentos comprobatórios, bem como suas especificações e justificativas para utilização destes gastos, com o preenchimento e a apresentação do Relatório de Execução Financeira do Projeto – REFP e toda documentação necessária para controle interno da \textbf{COOPERADA(S)} e da \textbf{PROPONENTE} e exposta nos módulos do \textbf{PROPDI} e do Manual de Auditoria ANEEL vigente a época de assinatura deste \textbf{CONTRATO} em consonância dos termos disposto no § único, art. 70, da CRFB/88, art. 93, do Decreto Lei 200/67 e art. 66, do Decreto nº 93.872/86, demonstrando a correta e regular aplicação dos recursos sob a responsabilidade da(s) \textbf{EXECUTORA(S)} como meio de se atingir o objetivo do \textbf{PROJETO};

\xx Manter toda documentação comprobatória das despesas realizadas referentes ao \textbf{PROJETO} à disposição da \textbf{COOPERADA(S)} quando esta financiar o recurso) da documentação comprobatória, técnica e financeira, entregue pela(s) \textbf{EXECUTORA(S)}. Os documentos devem ser classificados e apresentados da seguinte maneira:

\xxx  As despesas com materiais de consumo (rubrica MC) serão comprovadas mediante apresentação das notas fiscais de aquisição à \textbf{COOPERADA(S)} e/ou a \textbf{PROPONENTE} a nota fiscal geral da aquisição, apresentando uma declaração, que deverá conter a descrição individual dos materiais de consumo adquiridos para o \textbf{PROJETO} em referência, sua quantidade, preço unitário e total, além da identificação cadastral completa do fornecedor do material;

\xxx  Despesas referentes a materiais permanentes ou equipamentos (rubrica MP) serão comprovadas por meio da nota fiscal emitida em nome da(s) \textbf{EXECUTORA(S)}, pelos fornecedores dos materiais e comprovação da anuência previa da ANEEL.

\xxxx Ressalta-se que os bens inventariáveis (materiais permanentes ou equipamentos) só poderão ser adquiridos por entidade EXECUTORA com fins lucrativos após comprovação da anuência previa da ANEEL e deverão apresentar, no mínimo, 3 (três) cotações do produto.

\xxxx Bens inventariáveis (materiais permanentes ou equipamentos) poderão ser adquiridos por entidade EXECUTORA pública ou privada sem fins lucrativos e comporão seu patrimônio, sem necessidade de anuência da ANEEL para esta destinação desde que apresente, no mínimo, 3 (três) cotações do produto.

\xxxx Quando a aquisição dos materiais permanentes ou equipamentos destinados ao \textbf{PROJETO} for realizada pela \textbf{COOPERADA(S)} e/ou pela \textbf{PROPONENTE}, a(s) \textbf{EXECUTORA(S)} se comprometem a enviar a especificação técnica que atenda os procedimentos de compra da \textbf{COOPERADA(S)} e/ou a \textbf{PROPONENTE}, de forma que a \textbf{COOPERADA(S)} e/ou a \textbf{PROPONENTE} execute a compra após a aprovação desta especificação técnica e consulta de fornecedores no mercado;

\xxx As despesas com viagens e diárias (rubrica VD) tais como, passagens aéreas, taxas de embarque, diárias em hotéis (hospedagem e refeições), táxis, locação ou uso de veículos, serão comprovados mediante apresentação das respectivas notas fiscais/débito/recibos e/ou outro tipo de documento que comprove inequivocamente o gasto realizado, juntamente com o Relatório de Viagens em formato preestabelecido pela \textbf{COOPERADA(S)} e pela \textbf{PROPONENTE} e onde constarão todas as justificativas, até os limites por cada dia de viagem, estabelecidos abaixo:

% lista

\begin{itemize}[leftmargin=3cm]
    \item Hospedagem
          \begin{itemize}
              \item Cidades Brasileiras R\$ 350,00
              \item América do Sul / Central e México US\$ 250,00
              \item Estados Unidos e Canadá US\$ 400,00
              \item Europa e Demais países US\$ 400,00
          \end{itemize}
    \item Refeição
          \begin{itemize}
              \item Brasil (Capitais) R\$ 100,00
              \item Brasil (Não Capitais) R\$ 75,00
              \item América do Sul e Central US\$ 100,00
              \item América do Norte US\$ 150,00
              \item Europa US\$ 150,00
          \end{itemize}
\end{itemize}

PARÁGRAFO ÚNICO: Para cada despesa de viagem deve ser indicado nos documentos de comprovação do REFP e no relatório de viagem, o nome do viajante, o destino, o motivo da viagem e o período.

\xxx  As despesas com serviços de terceiros (rubrica ST), tais como contratação de empresas para desenvolvimento de uma tarefa especifica e necessária à pesquisa e previamente descrita e justificada no PLANO DE TRABALHO, deve ser comprovada com a apresentação de, no mínimo, 3 (três) cotações do serviço e da nota fiscal de serviços emitida em nome da(s) \textbf{EXECUTORA(S)}.

\xxx  As despesas com recursos humanos (rubrica RH) devem ser comprovadas com a apresentação de uma declaração constando:

% lista
\begin{enumerate}[label=\alph*), leftmargin=3cm]
    \item Nome completo do recurso humano;
    \item CPF;
    \item Titularidade acadêmica;
    \item Função no \textbf{PROJETO};
    \item Custo por hora;
    \item Quantidade de horas por mês, limitadas a 160h/mês;
    \item Período, em meses, referente à remuneração de acordo com o cronograma do \textbf{PROJETO};
    \item Valor total;
    \item Assinatura da pessoa física recebedora do recurso financeiro;
    \item Assinatura do Coordenador do \textbf{PROJETO};
    \item Além da completa identificação cadastral da(s) entidade(s) \textbf{EXECUTORA(S)} que a pessoa física, membro da equipe neste \textbf{PROJETO} é legalmente vinculada;
    \item O número de horas trabalhadas no mês deve ser um número inteiro (não são permitidos valores fracionados de hora).

\end{enumerate}

\xxx  As despesas classificadas como outros (rubrica OU), conforme consta no \textbf{PROPDI}, devem ser comprovadas com a apresentação da nota fiscal emitida em nome da(s) \textbf{EXECUTORA(S)}, além de uma declaração, que deverá conter a descrição individual do serviço adquirido para o \textbf{PROJETO} em referência, sua quantidade, preço unitário e total, identificação cadastral completa do fornecedor do material/Serviço e as devidas justificativas.

\xxxx O custo relativo à taxa de administração e mobilização de infraestrutura existente da(s) \textbf{EXECUTORA(S)}, está limitado a 5\% (cinco por cento) do valor realizado pela EXECUTORA em cada etapa.

\xx Responsabilizar-se pela condução técnica da execução das atividades do \textbf{PROJETO} visando a sua implementação e o seu desenvolvimento, conforme detalhado no PLANO DE TRABALHO;

\xx Prestar apoio científico e tecnológico necessário ao melhor desempenho das etapas de pesquisa do \textbf{PROJETO};

\xx Encaminhar, formalmente, com a necessária antecedência, as informações, documentos, os recursos e os dados que se façam indispensáveis à adequada execução do \textbf{PROJETO};

\xx Autorizar a participação no \textbf{PROJETO} de seus professores, pesquisadores, demais servidores e ou funcionários, conforme relacionados no PLANO DE TRABALHO, nos termos da legislação aplicável;

\xx A(s) EXECUTORAS devem ressarcir integralmente a \textbf{PROPONENTE} e \textbf{COOPERADA(S)} de qualquer valor relativo a eventuais aplicações de glosas financeiras pela Agência Nacional de Energia Elétrica – ANEEL referentes aos valores repassados às \textbf{EXECUTORA(S)} no âmbito do \textbf{CONTRATO}, inclusive após o término da vigência do \textbf{CONTRATO};

\xx Permitir a utilização de seus laboratórios, equipamentos, instrumentos e materiais, quando necessárias à execução do \textbf{PROJETO};

\xx Indicar o coordenador encarregado das atividades de pesquisa do \textbf{PROJETO}, o qual, conjuntamente com a(s) \textbf{EXECUTORA(S)}, será responsável pelo acompanhamento e avaliação das atividades desenvolvidas no âmbito deste \textbf{CONTRATO};

\xx Acompanhar e avaliar o atendimento dos resultados esperados sobre a execução das atividades previstas no \textbf{PROJETO};

\xx Elaborar por meio do coordenador e da equipe do \textbf{PROJETO}, e entregar tempestivamente à \textbf{COOPERADA(S)} e à \textbf{PROPONENTE} toda documentação exigida pela ANEEL, para cada fase da CADEIA DE INOVAÇÃO, tais como, sem, contudo, se limitar a:
\begin{enumerate}[label=\alph*), leftmargin=2cm]
    \item Arquivo eletrônico XML do \textbf{PROJETO};
    \item Arquivo eletrônico XML de interesse de execução do \textbf{PROJETO};
    \item Arquivo eletrônico XML de início de execução do \textbf{PROJETO};
    \item Arquivo eletrônico XML de prorrogação da execução do \textbf{PROJETO};
    \item Arquivo eletrônico XML do relatório final do \textbf{PROJETO};
    \item Arquivo eletrônico XML do relatório auditoria do \textbf{PROJETO};
    \item Arquivo eletrônico PDF do relatório final do \textbf{PROJETO}, em formato definido pela \textbf{PROPONENTE};
    \item Todos os trabalhos acadêmicos e publicações realizadas durante o \textbf{PROJETO};
    \item Artigo Técnico no Modelo CITENEL para publicação no evento;
    \item Relatório final do \textbf{PROJETO} conforme \hyperlink{3.33}{SUBCLÁUSULA 3.33.}; e
    \item Declaração dos integrantes do \textbf{PROJETO}, indicando as alterações de equipe, se houver.
\end{enumerate}

\xx Cumprir fielmente com toda legislação e regulamentos pertinentes à execução do \textbf{PROJETO} objeto deste \textbf{CONTRATO}, em especial com o \textbf{PROPDI} e com todas as normas e convenções aplicáveis no Brasil que proíbem atos de corrupção e outros atos lesivos contra a administração pública, dentre elas a Convenção Anticorrupção da OCDE, a Convenção das Nações Unidas contra a Corrupção (Decreto Federal nº 5.687/06), o Código Penal Brasileiro (Decreto-Lei nº 2.848/1940), a Lei de Improbidade Administrativa (Lei nº 8.429/1992), a Lei n° 9.613/98 e a Lei n° 12.846/2013 (em conjunto, “Leis Anticorrupção”) e no Código de Ética da \textbf{COOPERADA(S)} e da \textbf{PROPONENTE} e nas Leis Anticorrupção.
A CONTRATADA obriga-se a não empregar e/ou utilizar e fazer com que os \textbf{SUBCONTRATADOS} e demais terceiros não empreguem e/ou utilizem mão de obra infantil, escrava e/ou análoga à escravidão para a execução do presente \textbf{CONTRATO}, durante todo seu prazo de vigência, bem como se obriga a não subcontratar e/ou manter relações negociais com quaisquer outras empresas que empreguem, explorem e/ou por qualquer outro meio ou forma, utilizem o trabalho infantil em inobservância ao contido na legislação que regulamenta a matéria;
A
\xx Comprometer-se e certificar que todo recurso humano alocado, parcial ou integralmente no \textbf{PROJETO}, não possui qualquer impedimento para cumprir as obrigações inerentes ao objeto deste \textbf{CONTRATO}, inclusive no tocante à disponibilidade de dedicação horária, admitindo a ausência de alocação em outros projetos ou de dedicação exclusiva em outras funções.

\xxx  A(s) \textbf{EXECUTORA(S)} deverão garantir formalmente, através de declaração, que todos os integrantes do referido \textbf{PROJETO}, possuem competência técnica e disponibilidade legal para a plena e irrestrita execução deste \textbf{PROJETO} nos termos firmados neste \textbf{CONTRATO} e no PLANO DE TRABALHO anexo;

\xxx  A(s) \textbf{EXECUTORA(S)} comprometem-se a entregar, junto com este \textbf{CONTRATO} assinado, um documento particular entre a(s) \textbf{EXECUTORA(S)} e, individualmente, com cada membro nominado no referido \textbf{PROJETO}, contendo:

\begin{enumerate}[label=\alph*), leftmargin=3cm]
    \item Nome completo do membro do referido \textbf{PROJETO};
    \item Número do CPF do membro do referido \textbf{PROJETO};
    \item Função do membro do referido \textbf{PROJETO};
    \item Formação acadêmica do membro do referido \textbf{PROJETO}, incluindo curso(s) e instituição;
    \item Titulação acadêmica do membro do referido \textbf{PROJETO};
    \item Assinatura do membro do referido \textbf{PROJETO};
    \item Endereço eletrônico do Currículo Lattes atualizado do membro do referido \textbf{PROJETO};
    \item Declaração, que o integrante, assinante do documento em questão, possui competência técnica e disponibilidade legal para a plena e irrestrita execução deste \textbf{PROJETO} nos termos firmados neste \textbf{CONTRATO} e no PLANO DE TRABALHO anexo, admitindo:
          \begin{enumerate}[label=\roman*.]
              \item Inexistência de alocação em outros projetos que, somados com este, extrapolem 160 horas de trabalho mensal;
              \item Inexistência de sobreposição da ocupação horária entre projetos ou atividades;
              \item Inexistência de dedicação exclusiva em outras funções ou outras atividades;
          \end{enumerate}

\end{enumerate}

% tem link
\xxx  A(s) \textbf{EXECUTORA(S)} comprometem-se a enviar a \textbf{COOPERADA(S)} e a \textbf{PROPONENTE} o documento citado na CLÁUSULA TERCEIRA, \hyperlink{3.31.2}{SUBCLÁUSULA 3.31.2.} acima sempre que houver alteração na equipe.

\xxx  No caso em que haja alteração de equipe, o novo membro deverá ter titulação igual ou superior ao membro substituído. A alteração deverá ser solicitada formalmente com o preenchimento de Formulário de Solicitação de Mudanças em Projetos, em formato pré-estabelecido pela \textbf{	COOPERADA(S)} e/ou pela \textbf{PROPONENTE}.

\xx A(s) \textbf{EXECUTORA(S)} garantem que estão enquadradas como instituição de pesquisa e desenvolvimento reconhecidas pelo Ministério da Ciência e Tecnologia e Inovações – MCTIou instituições de ensino superior credenciadas junto ao Ministério da Educação – MEC, sendo estes os tipos de empresa(s) \textbf{EXECUTORA(S)} de projetos de pesquisa, desenvolvimento e inovação permitidas pelo \textbf{PROPDI} e pela Lei 9.991/2000, afastando da \textbf{COOPERADA(S)} e da \textbf{PROPONENTE} qualquer irregularidade no atendimento à Lei nº 9.991/2000 e ao disposto no \textbf{PROPDI}.

\xx A(s) \textbf{EXECUTORA(S)} ficará(rão) responsável(is) por compilar em (XX) relatório(s) final (is), sendo um para cada fase da cadeia de inovação, as informações de todo o \textbf{PROJETO}, incluindo os dados de todas as empresas participantes do \textbf{PROJETO}.

\xxx  O relatório final será considerado como um dos produtos do \textbf{PROJETO} e deverá ser entregue ao final de cada fase da cadeia de inovação deste \textbf{CONTRATO} e conforme estabelecido no PLANO DE TRABALHO.

\xxx  O relatório final deverá conter em destaque, a menção ao Programa de PDI regulado pela ANEEL e à(s) empresa(s) de energia elétrica que deram suporte ao \textbf{PROJETO}, conforme determinado no \textbf{PROPDI}.

\xxx  O relatório final deverá seguir uma estrutura mínima e obrigatória, porém não exaustiva, considerando todos os itens e subitens abaixo listados:

{\itshape
\leftskip 2cm
\begin{lmarginbox}
    \begin{enumerate}[leftmargin=1cm, font=\bfseries]
        \setstretch{1}
        \item[1.1.] \textbf{IDENTIFICAÇÃO DETALHADA DOS PARTICIPES E RESPECTIVAS EQUIPES (COM CURRÍCULO DETALHADO)}

        \item[2.1.] \textbf{OBJETIVO(S) DO \textbf{PROJETO}}

        \item[2.2.] \textbf{PALAVRAS-CHAVE}

        \item[2.3.] \textbf{PRODUTOS}

            \begin{enumerate}[font=\bfseries]
                \item[2.3.1.] Descrição detalhada de todos os produtos alcançados, tanto parciais quanto finais, com respectivas demonstrações visuais e textuais, destacando a logomarca oficial “PDI ANEEL” nos equipamentos desenvolvidos e/ou que compõe os produtos.
            \end{enumerate}

        \item[3.1.]	\textbf{ESTUDO DE ANTERIORIDADE}

            \begin{enumerate}[font=\bfseries]
                \item[3.1.1.]	Descrever a qualidade e abrangência da revisão bibliográfica e do estudo de anterioridade da pesquisa do estado da arte apresentadas, destacando os seguintes tópicos:

                    \begin{enumerate}[font=\bfseries]
                        \item[a)] Listagem de projetos similares na base de PDI ANEEL;
                        \item[b)] Busca de patentes e de registro de software no Instituto Nacional da Propriedade Industrial - INPI;
                        \item[c)] Listagem de produtos similares disponíveis no mercado;
                        \item[d)] Listagem de metodologias correlatas publicadas em periódicos indexados internacionais e nacionais.
                    \end{enumerate}
            \end{enumerate}

        \item[3.2.] \textbf{CONTRIBUIÇÃO AO ESTADO DA ARTE}
            \begin{enumerate}[font=\bfseries]
                \item[3.2.1.] Descrever o ineditismo, a inovação e a contribuição ao estado da arte obtida pelo \textbf{PROJETO}, sendo observado o seu período de execução e fase na cadeia de inovação.
            \end{enumerate}

        \item[3.3.]	\textbf{ORIGINALIDADE DO PRODUTO OU TÉCNICA}
            \begin{enumerate}[font=\bfseries]
                \item[3.3.1.] Descrever o entendimento sobre o \textbf{PROJETO}, relacionando o produto, a técnica/metodologia e a fase da cadeia da inovação.
                \item[3.3.2.] Destacar a presença de componentes da originalidade e descrever o enquadramento da proposta ou do \textbf{PROJETO} como atividade de PDI.
                \item[3.3.3.] Justificar o enquadramento do \textbf{PROJETO} como Cabeça de Série ("CS") – TRL 7, Lote Pioneiro ("LP") – TRL 8 ou Inserção no Mercado ("IM") – TRL 9 e as comprovações de desenvolvimentos das fases anteriores, quando o \textbf{PROJETO} for inicialmente ou de fato enquadrado como CS (TRL 7), LP (TRL 8) ou IM (TRL 9).
                \item[3.3.4.] Justificar os produtos importados utilizados no \textbf{PROJETO} e como procurou-se desenvolver soluções nacionais substitutas.
                \item[3.3.5.] Justificar, em caso de não obtenção do produto proposto, destacando:
                    \begin{enumerate}[font=\bfseries]
                        \item[a)] A originalidade/inovação da metodologia empregada,
                        \item[b)] O mérito científico da pesquisa realizada,
                        \item[c)] O conhecimento gerado e sua contribuição para novas investigações ou desenvolvimentos.
                    \end{enumerate}
                \item[3.3.6.] Descrever no mínimo 2 (dois) dos cinco quesitos de originalidade do \textbf{PROJETO}, quais sejam:
                    \begin{enumerate}[font=\bfseries]
                        \item[a)] Inexistência de produto similar no mercado nacional;
                        \item[b)] Ineditismo da aplicação de metodologia, material ou procedimento;
                        \item[c)] Registro de patente ou de software;
                        \item[d)] Geração de metodologia ou produto inovador, inclusive os baseados em produções acadêmicas originais, incluindo teses de doutorado;
                        \item[e)] Publicações relacionadas ao produto/metodologia em periódicos internacionais e/ou nacionais classificados na lista Qualis Periódicos como A1, A2 ou B1 no ano de publicação do artigo.
                    \end{enumerate}
            \end{enumerate}

        \item[4.1.]	\textbf{ABRANGÊNCIA DA APLICAÇÃO (APLICABILIDADE)}
            \begin{enumerate}[font=\bfseries]
                \item[4.1.1.] Descrever o real potencial de adoção e utilização dos resultados do \textbf{PROJETO} e a extensão do campo de ação em que o produto ou técnica é aplicável, destacando os seguintes aspectos:
                    \begin{enumerate}[font=\bfseries]
                        \item[a)] Extensão: classificar os resultados como nicho de aplicação, utilidade para a \textbf{COOPERADA(S)} e \textbf{PROPONENTE}, para o setor elétrico ou aplicação geral;
                        \item[b)] Segmento: possibilidade de aplicação em diferentes segmentos do setor elétrico (geração, transmissão, distribuição);
                        \item[c)] Setor econômico: possibilidade de aplicação além do setor elétrico;
                        \item[d)] Classe de consumo: possibilidade de aplicação em benefício de diferentes classes de consumidores: residencial, comercial, industrial, rural, poder público, etc.
                        \item[e)] Número de consumidores: discorrer sobre a quantidade de consumidores a serem beneficiadas pela aplicação dos resultados;
                        \item[f)] Potenciais usuários: possibilidade de utilização em massa por empresas ou pessoas.
                    \end{enumerate}
            \end{enumerate}
        \item[4.2.]	\textbf{TESTES DE FUNCIONALIDADE}
            \begin{enumerate}[font=\bfseries]
                \item[4.2.1.] Descrever a metodologia empregada nos testes de funcionalidade, discorrer sobre seus resultados e a efetividade destes ensaios.
            \end{enumerate}

        \item[5.] \textbf{CONTRIBUIÇÕES E IMPACTOS} Descrever as contribuições e impactos do \textbf{PROJETO} em termos econômicos, tecnológicos, científicos e socioambientais, incluindo todos os seus resultados e considerando o tema do \textbf{PROJETO} e sua fase na cadeia de inovação.

        \item[5.1.]	\textbf{CONTRIBUIÇÕES E IMPACTOS ECONÔMICOS}
            \begin{enumerate}[font=\bfseries]
                \item[5.1.1.] Descrever o impacto econômico de acordo com os seguintes parâmetros:
                    \begin{enumerate}[font=\bfseries]
                        \item[a)] Produtividade: Descrever a decorrência de mudanças nos processos operacionais ou administrativos da empresa de energia elétrica, reduzindo custos de mão-de-obra, materiais, insumos e/ou tempo de execução das atividades;
                        \item[b)] Qualidade do Fornecimento: Descrever a melhoria nos serviços prestados pode ser avaliada pela melhoria dos índices de satisfação e de qualidade da energia fornecida;
                        \item[c)] Gestão de Ativos: Descrever os ganhos econômicos, que podem ser decorrentes da redução ou da postergação de investimentos na expansão ou manutenção do sistema elétrico, bem como da redução de perdas não técnicas e comerciais, e do índice de furto de equipamentos ou materiais;
                        \item[d)] Mercado da Empresa: Descrever o impacto no mercado de energia da empresa e de outras empresas do setor, de forma a reduzir o custo da energia gerada, adquirida ou transmitida, e/ou os erros de previsão do mercado futuro de energia elétrica;
                        \item[e)] Eficiência Energética: Descrever os ganhos econômicos decorrentes da melhoria da eficiência energética na oferta de energia (geração, transmissão e distribuição) ou no uso final. No lado da oferta, pode ser decorrência de aumento na eficiência do sistema de geração, transmissão e/ou distribuição de energia. Do lado da demanda, pode ser decorrência de aumento na eficiência dos equipamentos de uso final, ao economizar de energia (kWh) ou reduzir demanda no horário de ponta do sistema (kW);
                        \item[f)] Outros: podem ser apresentados outros parâmetros que a(s) \textbf{EXECUTORA(S)} julgue(m) conveniente(s), desde que identificados os respectivos benefícios econômicos.
                    \end{enumerate}
            \end{enumerate}

        \item[5.2.]	\textbf{CONTRIBUIÇÕES E IMPACTOS TECNOLÓGICOS}
            \begin{enumerate}[font=\bfseries]
                \item[5.2.1.] Descrever o impacto tecnológico, considerando o apoio à infraestrutura laboratorial, a propriedade intelectual e os cursos de capacitação profissional.
                \item[5.2.2.] Descrever o apoio à infraestrutura laboratorial com base na aquisição de materiais permanentes e equipamentos para a execução do \textbf{PROJETO}, considerando a realidade da entidade beneficiada e os seguintes tópicos:
                    \begin{enumerate}[font=\bfseries]
                        \item[a)] Materiais permanentes e equipamentos, identificação do laboratório (novo ou existente) e a área de pesquisa;
                        \item[b)] Doação/cessão de bens para as entidades \textbf{EXECUTORA(S)}, caso haja.

                    \end{enumerate}
                \item[5.2.3.] Descrever a propriedade intelectual, destacando o tipo de registro de propriedade, o número do pedido/registro, a data e local de depósito/registro, o título, o nome do depositante e o nome do inventor, classificando conforme as seguintes definições do INPI:
                    \begin{enumerate}[font=\bfseries]
                        \item[a)] Patente de Invenção: avanços do conhecimento técnico que combinem novidade, atividade inventiva e aplicação industrial;
                        \item[b)] Patente de Modelo de Utilidade: objeto de uso prático, susceptível de aplicação industrial, que apresente nova forma ou disposição, envolvendo ato inventivo, que resulte em melhoria funcional no seu uso ou em sua fabricação;
                        \item[c)] Registro de Software: direito de propriedade sobre software;
                        \item[d)] Registro de Desenho Industrial: direito de propriedade sobre desenho industrial.
                    \end{enumerate}
                \item[5.2.4.] Descrever sobre os cursos de capacitação profissional realizados no \textbf{PROJETO}, classificando de acordo com as seguintes definições:
                    \begin{enumerate}[font=\bfseries]
                        \item[a)] Pós-graduação lato sensu;
                        \item[b)] Cursos técnicos;
                        \item[c)] Cursos de treinamento.
                    \end{enumerate}

            \end{enumerate}

        \item[5.3.]	\textbf{CONTRIBUIÇÕES CIENTÍFICAS (RELEVÂNCIA)}
            \begin{enumerate}[font=\bfseries]
                \item[5.3.7.] Descrever o impacto científico do \textbf{PROJETO}, com base nos seguintes itens:
                    \begin{enumerate}[font=\bfseries]
                        \item[a)] no tipo de produção técnico-científica (Periódico ou Anais; Nacional ou Internacional), o título do trabalho, o nome do periódico, a classificação Qualis da Coordenadoria de Aperfeiçoamento de Pessoal de Nível Superior – CAPES na data de publicação, o nome do evento e a cidade onde foi realizado;
                        \item[b)] nos cursos de pós-graduação realizados no período de execução do \textbf{PROJETO}, considerando o tema do \textbf{PROJETO}, o reconhecimento da instituição pelo Ministério da Educação – MEC e a recomendação da CAPES, averiguando as instituições, a quantidade e os tipos de cursos realizados, as datas de conclusão, os nomes dos membros da equipe, os diplomas, certificados, declarações e/ou histórico escolar.
                    \end{enumerate}
            \end{enumerate}

        \item[5.4.]	\textbf{CONTRIBUIÇÕES E IMPACTOS SOCIOAMBIENTAIS}
            \begin{enumerate}[font=\bfseries]
                \item[5.4.1.] Descrever os impactos socioambientais, considerando os benefícios e/ou danos ao meio ambiente e à sociedade, observando os seguintes tópicos:
                    \begin{enumerate}[font=\bfseries]
                        \item[a)] Riscos e impactos ambientais, considerando o meio físico, biológico e ecossistemas naturais: análise da vulnerabilidade, sensibilidade e mitigação sobre os fatores naturais envolvidos, como o subsolo, o solo, as águas, o ar e o clima, bem como sobre a fauna e a flora;
                        \item[b)] Riscos e impactos sociais, considerando saúde, segurança e o bem-estar da população e medidas de proteção: impactos na segurança ou qualidade de vida da comunidade e comprometimentos sobre o patrimônio cultural;
                        \item[c)] Impactos socioeconômicos: desenvolvimento de novas atividades socioeconômicas e geração de renda e/ou emprego;
                        \item[d)] Divulgação de informações e engajamento de partes interessadas.

                    \end{enumerate}

            \end{enumerate}

        \item[6.1.]	\textbf{ETAPAS E CRONOGRAMA DE EXECUÇÃO}
            \begin{enumerate}[font=\bfseries]
                \item[6.1.1.] Especificação, necessidade e justificativa de cada uma das alterações do \textbf{PROJETO}, sejam elas de tempo, recursos ou valores;

            \end{enumerate}

        \item[6.2.]	\textbf{RECURSOS EMPREGADOS, JUSTIFICATIVAS E RAZOABILIDADE DOS CUSTOS}
            \begin{enumerate}[font=\bfseries]

                \item[6.2.1.] Especificação de cada gasto realizado na execução do \textbf{PROJETO}, com descrição do objeto, da necessidade do uso bem como sua comprovação fiscal, respectiva rubrica e etapa, conforme determinado no modulo 2, item 2.1.7 do PROP\&D;
                    \begin{enumerate}[font=\bfseries]
                        \item[6.2.1.1.] Destacar a contratação de pesquisador estrangeiro no \textbf{PROJETO}, caso ocorra;
                        \item[6.2.1.2.]	Destacar as despesas com a taxa de administração e os custos relativos à mobilização de infraestrutura existente da(s) \textbf{EXECUTORA(S)}, caso ocorra;
                        \item[6.2.1.3.]	Destacar as despesas com construção, ampliação, reforma, adequação/montagem de laboratórios, caso ocorra;
                        \item[6.2.1.4.]	Destacar as despesas com estudo(s) de mercado, com vistas à produção industrial ou à comercialização, caso ocorra;
                        \item[6.2.1.5.]	Destacar as despesas destinadas à realização de cursos de pós-graduação, caso ocorra;
                    \end{enumerate}
                \item[6.2.3.]	Declaração assumindo, sob as penas da lei, que os documentos de comprovações fiscais de cada gasto realizado na execução do \textbf{PROJETO} são cópias fieis e que os originais ficarão disponíveis para consulta pelos cinco anos subsequentes à data do parecer da ANEEL sobre o reconhecimento dos gastos;
                \item[6.2.4.]	Demonstração do balizamento pela média de preços praticada na região onde o \textbf{PROJETO} é executado dos custos realizados no \textbf{PROJETO};
                    \begin{enumerate}[font=\bfseries]
                        \item[6.2.4.1.]	Materiais e equipamentos que não contam com fornecedores locais devem ser balizados pela média de preço praticada pelo mercado nacional;
                        \item[6.2.4.2.]	Para os itens que não estejam disponíveis em território nacional, o balizamento deve ser feito pelo mercado internacional;
                    \end{enumerate}
            \end{enumerate}


        \item[6.3.]	\textbf{ESTUDO DE VIABILIDADE ECONÔMICA}
            \begin{enumerate}[font=\bfseries]
                \item[6.3.1.]Descrever detalhadamente os seguintes estudos econômicos do \textbf{PROJETO}, Estudo de Viabilidade Econômica (“EVE”), Tempo de Retorno do Investimento (“payback”), Taxa Interna de Retorno (“TIR”) e Valor Presente Líquido (“VPL”), tomando-se como referência os custos de execução do \textbf{PROJETO}, os custos de implantação dos resultados e os benefícios financeiros de sua aplicação. Para a fase de Pesquisa Básica, a apresentação dos estudos econômicos é opcional.
                \begin{enumerate}[font=\bfseries]
                    \item[6.3.2.1.]	Na eventualidade dos estudos econômicos indicarem a inviabilidade econômica ou financeira do \textbf{PROJETO} e/ou produto, apontar justificativas que validem a execução do \textbf{PROJETO}, bem como as estratégias possíveis, perspectivas de viabilização e desenvolvimentos futuros.
                \end{enumerate}
            \end{enumerate}

        \item[7.1.] \textbf{PEDIDOS DE CESSÃO E DOAÇÃO DE BENS}

        \item[7.2.] \textbf{DETALHAMENTO DOS BENS INVENTARIÁVEIS}
            \begin{enumerate}[font=\bfseries]
                \item[7.2.1.] Detalhamento e justificativa da necessidade de todos os bens inventariáveis adquiridos por entidade EXECUTORA pública ou privada sem fins lucrativos.
                \item[7.2.2.] Detalhamento e justificativa da necessidade e comprovação da anuência previa da ANEEL de todos os bens inventariáveis adquiridos por entidade EXECUTORA pública ou privada com fins lucrativos.
            \end{enumerate}

        \item[7.3.] \textbf{ANEXOS}
            \begin{itemize}
                \item Manuais técnicos;
                \item Publicações, artigos, monografias, dissertações, tese (Ressalta-se que o(s) autor(es) da(s) publicações, artigos, monografia(s), dissertação(ões) ou tese(s) deve(m) ser membro(s) da equipe do \textbf{PROJETO} e nominalmente identificado(s)).
                \item Artigo técnico modelo CITENEL
                \item Cópia da(s) fotografias, apresentação(ões) e material gráfico da(s) audiência(s) pública(s) dos resultados(s) e/ou outros eventos vinculada(s) ao tema/assunto do \textbf{PROJETO};
                \item Comprovante de deposito no INPI;
                \item Comprovações fiscais e financeiras;
                \item Outras informações complementares e pertinentes

            \end{itemize}
    \end{enumerate}
\end{lmarginbox}
}




\xx A EXECUTORA deverá realizar a compilação dos dados e confecção dos Relatórios Trimestrais e Anuais exigidos pelo \textbf{PROPDI}, entregá-los 30 (trinta) dias antes do prazo definido pela ANEEL, na periodicidade exigida pelo \textbf{PROPDI}, e no formato a ser definido pela \textbf{PROPONENTE}.

\xxx  O Relatório Trimestral deverá seguir uma estrutura mínima e obrigatória, considerando, e não se limitando a estes, os itens e subitens abaixo listados:

\begin{enumerate}[label=\alph*), leftmargin=3cm]
    \item Situação do projeto, descrição resumida das atividades; informações financeiras.
\end{enumerate}


\xxx  O Relatório Anual deverá seguir uma estrutura mínima e obrigatória, considerando, e não se limitando a estes, os itens e subitens abaixo listados:

\begin{enumerate}[label=\alph*), leftmargin=3cm]
    \item Descrição detalhada das atividades do último ano, descrição do produto no seu presente estágio de desenvolvimento;
    \item Dados mensais Financeiros, Engajamento de Indústrias (%), Alocação e Remuneração de RH;
    \item Resultados Tecnológicos: Produtos gerados (Especif. dos Produtos, TRL, Tipo e Status de Utilização, Anterioridade); Propriedade Intelectual (Código do Pedido, Escritório, Tipo de PI, Status);
    \item Resultados Econômicos: Licenciamento de Produtos; Geração Direta de Empregos, Melhoria da Qualidade do Serviço (Indicadores de Qualidade de Fornecimento);
    \item Resultados Acadêmicos: Produção Técnico-Científica (título, Qualis, DOI); Capacitação de Pessoas (Tipo, Instituição, Trabalho de Conclusão); Apoio à Infraestrutura Laboratorial (Identificação do laboratório, investimento);
    \item Resultados Socioambientais: Aumento de Eficiência Energética, Redução da Emissão de GHG; Créditos de Carbono; Redução de Resíduos; Melhoria de Qualidade do Ar, Água e Solo, Universalização/Inclusão Social.

\end{enumerate}

\xx Conforme consta em regra explícita no \textbf{PROPDI}, fica acordado que independentemente do conceito geral do \textbf{PROJETO}, itens de custo não especificados e/ou justificados no Relatório Final, ou que possuam especificações ou justificativas insuficientes, serão glosados individualmente.

\xx Adicionalmente aos itens acima descritos, destaca-se que Relatório Final deverá seguir uma estrutura mínima e obrigatória, porém não exaustiva, conforme consta na “Tabela 3 - Estrutura e conteúdo mínimo do Relatório Final em PDF” constante no \textbf{PROPDI};

\xx Até 7 dias antes da assinatura deste contrato e do início do \textbf{PROJETO}, a(s) \textbf{EXECUTORA(S)} deve(m) apresentar o resultado da busca de anterioridade, executada por empresa plenamente capacitada e independente, atestando a originalidade do projeto. A busca de anterioridade deverá ser realizada no mínimo nas seguintes bases e sem se limitar a: Anais de eventos do setor elétrico tais como Citenel, SENDI, SNPTEE e outros; Bancos de publicações de produção científica tais como SCielo, Engeneering Village, Scopus, e outros; Bancos de patentes e registros tais como INPI e USPTO.  O Relatório de busca de anterioridade deverá constar de forma explícita a evidência de originalidade do projeto e a negativa da existência da anterioridade dos produtos finais do projeto. Este resultado deve integrar de forma sintética o XML do \textbf{PROJETO}, no campo Originalidade.

\xx Antes do início do \textbf{PROJETO}, a(s) \textbf{EXECUTORA(S)} devem apresentar um cronograma de execução preenchido e atualizado, em formato MSProject, contendo os recursos humanos, financeiros e demais recursos realização do projeto conforme modelo fornecido pela \textbf{COOPERADA(S)} e/ou pela \textbf{PROPONENTE}.

\xx Sempre que ocorrer desenvolvimento de software no âmbito do \textbf{PROJETO}, as \textbf{EXECUTORA(S)} se comprometem a:

\xxx  Desenvolver e atualizar durante a execução do projeto, o código fonte no repositório git indicado pela \textbf{COOPERADA(S)} e pela \textbf{PROPONENTE}.

\xxx  Entregar e manter atualizado, durante a execução do projeto, manual de instalação, manual de utilização, código fonte e software compilado.

\xxx  Realizar uma auditoria de todo código-fonte e do(s) software(s) compilado(s), apresentando um laudo conclusivo, sem nenhum apontamento de falha. Este laudo deverá ser emitido por uma empresa independente, que não possua vínculos direto ou indireto com a(s) \textbf{EXECUTORA(S)}. Este laudo deve validar positivamente os seguintes itens:

\xxxx Compilação sem erro do software a partir do código-fonte fornecido;

\xxxx Teste de segurança e ausência de vulnerabilidades conhecidas no código-fonte e no software compilado;

\xxxx Rastreio e ausência dos limites e pontos de interação com outros processos ou usuários não especificados na documentação;

\xxxx Ausência de “estouro de Buffer”, “Condição de corrida”, “Validações de Entradas” e outras falhas de programação comumente conhecidas;

\xxxx Tratamento de erros da aplicação;

\xxxx Procura por assinaturas de funções;

\xxxx Utilização de memória;

\xxxx Formato, padronização e legibilidade do código;

\xxxx Lista de funções e variáveis;

\xxxx Validação da consonância da documentação frente à aplicação.

\xx A EXECUTORA compromete-se a obedecer as regras a seguir, definidas visando o perfeito funcionamento do software, sendo expressamente proibida a inclusão de conteúdos (i) pornográficos, racistas ou ofensivos, (ii) ilícitos, (iii) criptografados ou protegidos por senha que contenham informações impróprias e/ou ilegais, (iv) calunioso, (v) difamatório, (vi) piratas, (vii) protegido por direitos autorais, sendo vedada a publicação de fotos, textos ou arquivos de áudio/som sem a autorização do representante da obra ou empresa responsável.

\xxx  Não obstante, sempre que houver desenvolvimento de software no âmbito deste \textbf{CONTRATO}, a(s) EXECUTORA(s) se compromete(m) a atender todos os padrões de desenvolvimento e entrega de softwares contidos no endereço: \url{https://github.com/taesa-tec/8000-Development-Pattern-Docs/wiki}.

\xx A EXECUTORA declara possuir pleno direito de uso de todo software, código, base de dados, imagens, bibliotecas, subprogramas, serviços e qualquer código inserido nos softwares desenvolvidos no âmbito deste \textbf{PROJETO}.

\xxx  A EXECUTORA tem o direito de ceder e cede plenamente à \textbf{COOPERADA(S)} e à \textbf{PROPONENTE} o uso de todo código e todo software desenvolvido, podendo a \textbf{COOPERADA(S)} e a \textbf{PROPONENTE} copiar, reproduzir, alterar e mudar o local de instalação, sem prévia autorização do autor ou de nenhuma outra PARTE ou entidade.

\xx Objetivando a construção de pequenas soluções tecnológicas, disponibilizar para a \textbf{COOPERADA(S)} e \textbf{PROPONENTE} e a exclusivo custo da executora, 40 horas por mês de: 1 desenvolvedor full-stack, com habilidades de desenvolvimento frontend e backend, e 40 horas por mês de: 1 designer de interface de usuário e experiência de usuário (UI/UX).

\xx A(s) EXECUTORA(s) compromete(m)-se a tratar as informações classificadas legalmente como dados pessoais em observância à legislação aplicável, inclusive, mas não se limitando à Lei Federal nº 13.709/2018 (“LGPD”) e a qualquer regulamentação complementar (“Leis de Proteção de Dados”), tanto no armazenamento ou processamento de informações classificadas legalmente como dados pessoais quanto na garantia de que seus desenvolvimentos de softwares, soluções e/ou sistemas são plenamente capazes de atender integralmente a referida lei e respectivas regulamentações.

\xx PARAGRAFO ÚNICO. A(s) EXECUTORA(s) garante que, no momento da entrega do(s) PRODUTO(S), este(s) está(ão) em pleno cumprimento à LGPD, sendo ela responsável por manter a \textbf{PROPONENTE} e COOPERADA indenes a qualquer reclamação que venha a sofrer em razão desta norma. \textbf{COOPERADA(S)}.

\xx A EXECUTORA, por seus administradores e procuradores, se compromete(m) a obter o consentimento e realizar o tratamento legítimo dos dados pessoais fornecidos por e/ou obtidos em seu nome, para que possa dar cumprimento ao presente \textbf{CONTRATO}, às obrigações legais e/ou regulatórias, tratando estes dados pessoais apenas no limite do necessário para o bom e fiel cumprimento de tais obrigações, observando e cumprindo a legislação de proteção de dados pessoais aplicável à presente relação contratual , estando expressamente proibido o uso dos dados pessoais para fins distintos.

\xx Nos termos da Lei nº 13.709/2018, a(s) EXECUTORA(s) será(ão) responsável(eis) por responder e satisfazer todas e quaisquer solicitações/reclamações advindas do titular dos dados pessoais e da ANPD (Autoridade Nacional de Proteção de Dados), que estejam relacionadas com o tratamento de tais dados, devendo, nesta hipótese, a outra PARTE colaborar com aquela, fornecendo as informações relativas ao tratamento realizado, para que a PARTE (Controladora) se manifeste.
