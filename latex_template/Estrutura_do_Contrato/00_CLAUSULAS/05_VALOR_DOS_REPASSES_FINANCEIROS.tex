\causula{CLÁUSULA QUINTA - VALOR DOS REPASSES FINANCEIROS}

\xx Para a execução do \textbf{PROJETO}, a \textbf{COOPERADA(S)} e a \textbf{PROPONENTE} repassarão à(s) \textbf{EXECUTORA(S)} a importância total de R\$ {\ValorContrato} (XXXX), ao longo dos 24 (vinte e quatro) meses de duração do \textbf{PROJETO} divididos de acordo com os valores de repasses de cada etapa previstos no \textbf{PLANO DE TRABALHO}, e observando especialmente o disposto na \hyperlink{6.2.1}{SUBCLÁUSULA 6.2.1}, partilhados da seguinte maneira:

\begin{itemize}[leftmargin=1cm]
    \item A \textbf{PROPONENTE} repassará a \NomeExecutoraA o valor de R\$ \RepasseProponenteExecutoraA (XXXX);
    \item A \textbf{PROPONENTE} repassará a \NomeExecutoraB o valor de R\$ \RepasseProponenteExecutoraB (XXXX);
    \item A \textbf{COOPERADA(S)} repassará a \NomeExecutoraA o valor de R\$ \RepasseCooperadaExecutoraA (XXXX); e
    \item A \textbf{COOPERADA(S)} repassará a \NomeExecutoraB o valor de R\$ \RepasseCooperadaExecutoraB (XXXX).
\end{itemize}

\xx Na importância acima mencionada, repassada à(s) \textbf{EXECUTORA(S)} pela \textbf{COOPERADA(S)} e pela \textbf{PROPONENTE}, estão inclusas todas as despesas inerentes à execução do PROJETO, tais como, sem, contudo, se limitar a, deslocamento, transporte, hospedagem, alimentação, recursos humanos, compra, uso e locação de equipamentos, materiais de consumo, uso de laboratórios, impostos, taxas, contribuições e quaisquer outros encargos cabíveis.

\xx Ao final da vigência do \textbf{CONTRATO} e/ou da execução do \textbf{PROJETO}, caso existam recursos financeiros não aplicados no \textbf{PROJETO} ou não comprovados, incluindo seus rendimentos, a(s) \textbf{EXECUTORA(S)} deverão informar à \textbf{COOPERADA(S)} e à \textbf{PROPONENTE} o valor final de cobrança minorado do resíduo total disponível na conta do \textbf{PROJETO} antes da emissão do último documento de cobrança, com a apresentação dos comprovantes.

\xxx No caso em que o repasse financeiro tenha sido realizado em etapas anteriores, sem que a \textbf{EXECUTORA} tenha utilizado corretamente o recurso para desenvolvimento do \textbf{PROJETO} ou não o tenha comprovado e, ao final da vigência do \textbf{CONTRATO} e/ou da execução do \textbf{PROJETO} ainda haja recurso não utilizado, deverá a \textbf{EXECUTORA} proceder à devolução do valor não empregado à \textbf{COOPERADA(S)} e à \textbf{PROPONENTE}, via emissão de nota de débito.
