\causula{CLÁUSULA DÉCIMA - DA CESSÃO}


\xx A(s) \textbf{EXECUTORA(S)} não poderão ceder a terceiros, bem como alterar ou subcontratar o escopo do presente \textbf{CONTRATO} e seus anexos, sem a prévia e expressa anuência da \textbf{COOPERADA(S)} e da \textbf{PROPONENTE}.

\xx A(s) \textbf{EXECUTORA(S)} terão total responsabilidade pelos atos e omissões praticados por seus cessionários e/ou \textbf{SUBCONTRATADOS}, devendo exigir destes a execução do objeto deste \textbf{CONTRATO} de acordo com os parâmetros técnicos e de qualidade fixados neste instrumento, observando, ainda, os prazos para execução das obrigações definidas neste \textbf{CONTRATO}.

\xx Se a \textbf{COOPERADA(S)} e/ou a \textbf{PROPONENTE} julgar que determinado cessionário e/ou \textbf{SUBCONTRATADO}, por qualquer razão, pode prejudicar a execução do \textbf{PROJETO}, ou se apresentar indícios de insolvência, poderá solicitar por escrito a sua substituição à(s) \textbf{EXECUTORA(S)}.

\xx A(s) \textbf{EXECUTORA(S)} terão 10 (dez) dias, a contar do recebimento da notificação elaborada pela \textbf{COOPERADA(S)} e/ou pela \textbf{PROPONENTE} para substituir o dito cessionário e/ou \textbf{SUBCONTRATADO}.

\xx Os empregados e funcionários dos cessionários e/ou \textbf{SUBCONTRATADOS} da(s) \textbf{EXECUTORA(S)} não dispõem de qualquer vínculo empregatício com a \textbf{COOPERADA(S)} e/ou a \textbf{PROPONENTE}. Não há, portanto, nenhum dos elementos caracterizadores do referido vínculo entre as mesmas, prestadoras do serviço público de transmissão de energia elétrica, e os empregados e funcionários das empresas encarregadas da execução dos serviços que constituem o \textbf{PROJETO} objeto deste \textbf{CONTRATO}.

\xx A(s) \textbf{EXECUTORA(S)} e seus cessionários e/ou \textbf{SUBCONTRATADOS} constituem empresas independentes, não existindo qualquer vínculo societário entre elas e a \textbf{COOPERADA(S)} e/ou a \textbf{PROPONENTE}.