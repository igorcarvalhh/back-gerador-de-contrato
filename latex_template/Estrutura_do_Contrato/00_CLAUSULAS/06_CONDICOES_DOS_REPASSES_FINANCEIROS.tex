\causula{CLÁUSULA SEXTA - CONDIÇÕES DOS REPASSES FINANCEIROS}

\xx O valor mencionado na \hyperlink{5.1}{SUBCLÁUSULA 5.1} acima será repassado à(s) \textbf{EXECUTORA(S)} pela \textbf{COOPERADA(S)} e pela \textbf{PROPONENTE} mediante a emissão de documentos de cobrança, de acordo com o critério estabelecido nesta cláusula.

\xx Os repasses financeiros do \textbf{PROJETO} serão efetuados no início de cada etapa, mediante apresentação pela(s) \textbf{EXECUTORA(S)} e aprovação, pela \textbf{PROPONENTE} (e pela \textbf{COOPERADA(S)} quando esta financiar a etapa) do RELATÓRIO DE ATIVIDADES E DESPESAS relativo às atividades anteriores ao repasse solicitado.

\xxx O valor dos repasses financeiros solicitados pela(s) \textbf{EXECUTORA(S)} somente poderão ocorrer com o resultado verdadeiro dos dois cálculos abaixo:

\newcommand{\dsum}{\displaystyle\sum}

\begin{enumerate}[label=\alph*), leftmargin=3cm]
    \everymath{\displaystyle}
    \item $ \left( 1- \left(  \frac{\dsum VC }{\dsum VR}  \right) \right) < 5\% $
    \item $ \left(\dsum VR- \dsum VC \right) < \dsum VP $

\end{enumerate}
Onde:

\begin{itemize}
    \item $\dsum VR$ representa a “Soma dos valores já repassados”;
    \item $\dsum VC$ representa a “Soma dos valores comprovados e aprovados no RELATÓRIO DE ATIVIDADES E DESPESAS”;
    \item $\dsum VP$ representa o “valor do próximo repasse financeiro”;

\end{itemize}

\xxx Os repasses financeiros do \textbf{PROJETO} serão pagos concomitantemente para toda(s) a(s) \textbf{EXECUTORA(S)} do \textbf{PROJETO}, após entrega (pelas \textbf{EXECUTORAS}) e aprovação (pela \textbf{PROPONENTE} e \textbf{COOPERADA(S)} quando esta financiar a etapa) de todas as obrigações da etapa de todas as \textbf{EXECUTORAS}.

\xxx Excetua-se da definição exposta na \hyperlink{6.2.1}{SUBCLÁUSULA 6.2.1} o primeiro repasse financeiro do \textbf{PROJETO}, ficando acordado entre os \textbf{PARTÍCIPES} que o primeiro repasse ocorrerá no primeiro período contábil após assinatura do presente \textbf{CONTRATO}.

\xxx Excetua-se da definição exposta nas \hyperlink{6.2}{SUBCLÁUSULA 6.2} e \hyperlink{6.2.1}{6.2.1} o último repasse financeiro do \textbf{PROJETO}, ficando acordado entre os PARTICIPES que o último repasse terá o montante mínimo de 10\% (dez por cento) do valor do \textbf{CONTRATO} e será realizado somente após a entrega e aprovação de toda documentação pendente e o cumprimento de todas as obrigações entre os \textbf{PARTÍCIPES}.

\xx O RELATÓRIO DE ATIVIDADES E DESPESAS deve ser composto pelos seguintes documentos:

\begin{enumerate}[label=\alph*), leftmargin=2cm]
    \item Relatório de Execução Financeira do \textbf{PROJETO} ("REFP");
    \item Respectivos comprovantes fiscais das despesas declaradas no REFP;
    \item Comprovante original e assinado dos recursos de Homem/Hora;
    \item Relatório(s) original(is) e assinado(s) das viagens e diárias (quando houver);
    \item Demonstrativos Financeiros (extratos bancários e razões contábeis) da conta remunerada exclusiva para o \textbf{PROJETO}, em formato digital; e
    \item Produto Técnico de execução da etapa aprovado pelo gerente do \textbf{PROJETO}.
\end{enumerate}

\xx Os Demonstrativos Financeiros (extratos bancários e razões contábeis) da conta remunerada exclusiva para o \textbf{PROJETO} deverão estar discriminados no REFP, e os recursos de tais rendimentos não devem ser utilizados pela(s) \textbf{EXECUTORA}(s) para custear despesas não previstas no PLANO DE TRABALHO, exceto sob expressa autorização da \textbf{COOPERADA(S)} e da \textbf{PROPONENTE}.

\xx As contrapartidas da(s) \textbf{EXECUTORA(S)}, caso haja, deverão estar discriminadas no REFP, e classificadas conforme a rubricas permitidas pela ANEEL no \textbf{PROJETO} de PDI, e a ele anexados os comprovantes fiscais.

\xx A \textbf{PROPONENTE} (e a \textbf{COOPERADA(S)} quando esta financiar a etapa) deve manifestar-se acerca da aprovação ou reprovação do RELATÓRIO DE ATIVIDADES E DESPESAS em até 30 (trinta) dias a contar de seu efetivo recebimento.

\xxx Em caso de não aprovação do RELATÓRIO DE ATIVIDADES E DESPESAS, a(s) \textbf{EXECUTORA(S)} deverão reapresentar o RELATÓRIO DE ATIVIDADES E DESPESAS de acordo com as exigências realizadas pela \textbf{PROPONENTE} (e a \textbf{COOPERADA(S)} quando esta financiar a etapa), dispondo a \textbf{PROPONENTE} (e a \textbf{COOPERADA(S)} quando esta financiar a etapa) de 30 (trinta) dias contar de seu efetivo recebimento para se manifestar sobre os RELATÓRIO DE ATIVIDADES E DESPESAS reapresentados pela(s) \textbf{EXECUTORA(S)}, e assim sucessivamente.

\xx Somente após a devida aprovação, pela \textbf{PROPONENTE} (e pela \textbf{COOPERADA(S)}, quando esta financiar a etapa), do RELATÓRIO DE ATIVIDADES E DESPESAS e dos produtos relacionados à conclusão da etapa a(s) \textbf{EXECUTORA(S)} poderão encaminhar o documento de cobrança contendo obrigatoriamente todos os dados abaixo, em qualquer situação sem que haja a aplicação de quaisquer penalidades, correção monetária e/ou juros à \textbf{PROPONENTE} e/ou \textbf{COOPERADA(S)}:

\begin{enumerate}[label=\alph*), leftmargin=2cm]
    \item Número da Nota Fiscal;
    \item Número do Pedido;
    \item Número do \textbf{CONTRATO};
    \item Nº e Título resumido do \textbf{PROJETO};
    \item Rubrica;
    \item Etapa do \textbf{PROJETO};
    \item Título da Etapa;
    \item Banco, agência e conta descrita neste \textbf{CONTRATO};
    \item Identificação de isenção ou imunidade tributária;
    \item Valor Total da Nota Fiscal.
\end{enumerate}

\xx A(s) \textbf{EXECUTORA(S)} deverão informar, no cadastro de fornecedor realizado no website da \textbf{PROPONENTE} e da \textbf{COOPERADA(S)}, uma conta remunerada, exclusiva para o \textbf{PROJETO}, em instituição financeira oficial, para recebimento dos recursos financeiros referentes à execução do \textbf{PROJETO} objeto deste \textbf{CONTRATO}. A conta informada no website da \textbf{PROPONENTE} e da \textbf{COOPERADA(S)} deve ser a mesma informada neste \textbf{CONTRATO} e deverá ter característica conservadora, não podendo ter rendimentos negativos. Os respectivos demonstrativos financeiros (extratos bancários e razões contábeis) deverão integrar a referida Prestação de Contas citada na CLÁUSULA TERCEIRA, SUBCLÁUSULA “3.13”.

\xx Os documentos de cobrança deverão ser encaminhados, pela(s) \textbf{EXECUTORA(S)}, somente após a devida aprovação, pela \textbf{PROPONENTE} e da \textbf{COOPERADA(S)}, do RELATÓRIO DE ATIVIDADES E DESPESAS e até o 15º (décimo quinto) dia do mês e antes dos prazos estabelecidos na cláusula sétima deste \textbf{CONTRATO}, para cada uma das etapas na cadeia de inovação. Caso a \textbf{EXECUTORA} deixe de observar este procedimento, a respectiva nota fiscal deve ser emitida no primeiro dia útil do mês subsequente, sem a aplicação de quaisquer penalidades, correção monetária e/ou juros à \textbf{COOPERADA(S)} e/ou \textbf{PROPONENTE}.

\xxx Os \textbf{PARTÍCIPES} concordam que a \textbf{COOPERADA(S)} e a \textbf{PROPONENTE} realizarão os repasses financeiros até as datas previstas na cláusula sétima deste \textbf{CONTRATO}, para cada uma das fases da cadeia de inovação. A \textbf{PROPONENTE} e a \textbf{COOPERADA(S)} não terão qualquer obrigação de realizar os repasses após estes prazos.

\xx A(s) \textbf{EXECUTORA(S)} emitirão tão somente notas fiscais, não sendo admissível a emissão de boletos bancários, duplicatas, ou qualquer outra modalidade de documento de cobrança sem a prévia e expressa aprovação por parte da \textbf{COOPERADA(S)} e da \textbf{PROPONENTE}.

\xxx As Notas Fiscais emitidas em decorrência da realização dos serviços deverão indicar o código fiscal 2.01 "Serviços de pesquisas e desenvolvimento de qualquer natureza", conforme lista de serviços anexa à Lei Complementar nº116 de 31 de julho de 2003. Para a \textbf{EXECUTORA} XXX, domiciliada no estado de São Paulo, o código fiscal correspondente será 03085, seguindo os preceitos do Anexo I da Instrução Normativa SF/SUREM nº8/2011 de 18 de julho de 2011.

\xxx O prestador de serviço ou \textbf{EXECUTORA(S)} que não estiver inscrito e em situação regular no Cadastro de Empresas Prestadoras de Outros Municípios (CEPOM) do Rio de Janeiro terá retenção de ISS conforme Resolução SMF Nº 2515, de 30 de julho de 2007.

\xxx O prestador de serviço ou \textbf{EXECUTORA(S)} que por qualquer motivo apresentar algum custo onde incide dupla tributação constatada por um ou mais comprovantes para o mesmo objeto de custo (viagem, diária, material permanente, material de consumo, serviço de terceiros, outros serviços ou materiais), deverá arcar com este custo com recursos próprios, não sendo a dupla tributação aprovada como custo do \textbf{PROJETO}.

\xx A(s) \textbf{EXECUTORA(S)} deverão emitir obrigatoriamente notas fiscais distintas para as atividades referentes a (1) Recursos Humanos (“RH”), (2) Materiais Permanentes e Equipamentos (“MP”), (3) Materiais de Consumo (“MC”), (4) Serviços de Terceiros (“ST”), (5) Viagens, Diárias (“VD”) e (6) Outros (“OU”).

\xxx Para os itens pertencentes à rubrica Outros (“OU”), deve-se emitir notas fiscais distintas para (i) as taxas de administração e custos relativos à mobilização de infraestrutura existente da(s) \textbf{EXECUTORA(S)} e ii) os demais custos da Rubrica OU.

\xx Os repasses financeiros referentes aos documentos de cobrança emitidos pela(s) \textbf{EXECUTORA(S)} serão efetuados em até 30 (trinta) dias contados da emissão da Nota Fiscal, desde que cumpridas todas as obrigações descritas neste \textbf{CONTRATO}, na seguinte instituição bancária indicada pela(s) \textbf{EXECUTORA(S)}, bastando o comprovante de depósito como documento hábil a comprovar a quitação do débito:

\leftskip 1cm
\textbf{\NomeExecutoraA}\\
BANCO: \BancoExeutoraA\\
AGÊNCIA: \AgenciaExecutoraA\\
CONTA CORRENTE: \ContaCorrenteExecutoraA\\
TITULARIDADE: \TitularidadeExecutoraA.

\textbf{\NomeExecutoraB}\\
BANCO: \BancoExeutoraB\\
AGÊNCIA: \AgenciaExecutoraB\
CONTA CORRENTE: \ContaCorrenteExecutoraB\\
TITULARIDADE: \TitularidadeExecutoraB.

\xxx A(s) \textbf{EXECUTORA(S)} declara(m) que a conta bancária informada na cláusula 6.10. acima é remunerada e exclusiva para os depósitos referentes aos repasses previstos neste \textbf{CONTRATO}, não sendo permitido seu uso para outros fins;

\xxx Os recursos referentes aos repasses realizados pela \textbf{COOPERADA(S)} e pela \textbf{PROPONENTE} à(s) \textbf{EXECUTORA(S)} devem permanecer na conta remunerada até sua efetiva utilização, não sendo permitida a transferência para outra conta bancária intermediária;

\xx Os documentos de cobrança deverão ser emitidos conforme as regras definidas nesta cláusula encaminhadas à sede da \textbf{PROPONENTE} (e pela \textbf{COOPERADA(S)} quando esta financiar a etapa), nos endereços definidos no preâmbulo do presente instrumento.

\xx O(s) repasse(s) financeiro(s) referente(s) ao último documento de cobrança para cada uma das fases do \textbf{PROJETO} à(s) \textbf{EXECUTORA(S)} estará(ão) condicionado(s) ao término de cada fase do \textbf{PROJETO}, conforme classificação na CADEIA DE INOVAÇÃO (“DE – TRL 5”, “CS – TRL 7”, “LP – TRL 8”, e “IM – TRL 9”) e à entrega e aprovação de todos os produtos relacionados ao \textbf{PROJETO}, bem como a apresentação do RELATÓRIO DE ATIVIDADES E DESPESAS final, relatório técnico final, artigo técnico modelo CITENEL e demais documentos descritos na SUBCLÁUSULA 3.28 deste \textbf{CONTRATO} para cada uma das três fases da CADEIA DE INOVAÇÃO contempladas no \textbf{PROJETO}.

\xx Caso qualquer das cobranças seja feita em desacordo com os ditames deste \textbf{CONTRATO} e seus anexos, os documentos de cobrança poderão ser recusados pela \textbf{PROPONENTE} (e pela \textbf{COOPERADA(S)} quando esta financiar a etapa), devendo estes documentos ser corrigidos e apresentados pela(s) \textbf{EXECUTORA(S)} nova solicitação de repasse financeiro, podendo a \textbf{PROPONENTE} (e \textbf{COOPERADA(S)} quando esta financiar a etapa) utilizar os prazos estabelecidos neste \textbf{CONTRATO} para aprovação e pagamento.

\xx A realização dos repasses financeiros não significa a sua aprovação definitiva do RADE pela \textbf{COOPERADA(S)} ou pela \textbf{PROPONENTE}. Todo repasse financeiro que vier a ser considerado indevido, seja por descumprimento de qualquer cláusula deste \textbf{CONTRATO} ou por descumprimento de qualquer item do \textbf{PROPDI}, será, a qualquer tempo, descontado dos repasses financeiros devidos à(s) \textbf{EXECUTORA(S)}, ou dela cobrado.

\xxx A \textbf{PROPONENTE} pode a qualquer tempo dentro da vigência deste \textbf{CONTRATO} identificar erros em qualquer entrega realizada pela(s) \textbf{EXECUTORA}(s) e solicitar a(s) \textbf{EXECUTORA}(s) a pronta correção em até 30 dias do informado. A correção do erro deve ser realizada mesmo que a referida entrega já tenha sido equivocadamente aprovada anteriormente pela \textbf{PROPONENTE}.

\xx Ocorrendo atraso no repasse de qualquer uma das parcelas, os valores serão corrigidos pela variação do IGP-M, pro rata die, acrescido de juros de mora de 1\% (um por cento) ao mês, no período entre o vencimento do documento de cobrança e a do efetivo repasse financeiro.

\xx A \textbf{EXECUTORA} renuncia expressamente à faculdade de emitir qualquer título de crédito em razão do presente \textbf{CONTRATO}, sendo vedado à \textbf{EXECUTORA} utilizar o presente \textbf{CONTRATO} em garantias de transações bancárias e/ou financeiras de qualquer espécie, bem como efetuar operação de desconto, negociar, repassar ou de qualquer forma ceder os créditos decorrentes da execução deste \textbf{CONTRATO} a instituições financeiras, empresas de “factoring” ou terceiros, sem a prévia e expressa aprovação por parte da \textbf{COOPERADA(S)} e/ou da \textbf{PROPONENTE}.
