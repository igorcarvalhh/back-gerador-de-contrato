%%%------------------------------------------------------------------------------%%
%%%	PAGE LAYOUT SPECIFICATIONS
%%%------------------------------------------------------------------------------%%


%%%--------------------------------------------------------------------------------
%%%	MARGINS SPECIFICATIONS
%%%--------------------------------------------------------------------------------
\usepackage{geometry} % Required to modify the page layout
\geometry{
    headheight=14cm,
    top=2.54cm,
    bottom=2.54cm,
    left=1.91cm,
    right=1.91cm}
    
%%%%%% ==============================  PACOTES  ============================== %%%%%%
\usepackage{xcolor}
\usepackage[T1]{fontenc}
\usepackage{hang}
\usepackage{amsmath,enumerate}
\usepackage{setspace}
\usepackage{tcolorbox}

%%% --- Listas e enumeradores ---%
\usepackage{hyperref}
\usepackage[sharp]{easylist}
\usepackage{mypackage}
\usepackage{enumitem}

%%%%  --- Layout de página ---%
\usepackage{fancyhdr}  % customized pages style

%%%  --- Gerador de Loren Ipsum ---%
\usepackage{lipsum}% juts to generate text for the example

%%%  --- paginação referencial "Pagina X de Y"  ---%
\usepackage{lastpage}

%%%  --- Permite caracteres acentuados ---%
\usepackage[utf8]{inputenc}

%%%%  --- Fontes  ---%
%\usepackage{newtxmath} % 
%\usepackage{mathptmx} % Use the Adobe Times Roman as the default text font together with math symbols from the Sym­bol, Chancery and Com­puter Modern fonts
%\usepackage{newtxtext} % 
%\usepackage{amsfonts} % Use de AMS font
\usepackage{avant} % Use the Avantgarde font 
\usepackage{microtype}
\usepackage{helvet}
\usepackage[normalem]{ulem}
\usepackage{soul}

%%%%  --- Tabelas ---%
%\usepackage{tabularx} % in the preamble
%\usepackage{showframe}% http://ctan.org/pkg/showframe
%\usepackage{booktabs}% http://ctan.org/pkg/booktabs
\usepackage{array}
\usepackage{graphicx}

%%%%  --- PDF Features  ---%
%\usepackage[author={P&D TAESA}]{pdfcomment}


%%%%%% ==============================  Configurações  ============================== %%%%%%

%%%--------------------------------------------------------------------------------
%%%	FONT SPECIFICATIONS
%%%--------------------------------------------------------------------------------

%Escolha da Fonte
\renewcommand{\familydefault}{\sfdefault}
%%%%\renewcommand{\rmdefault}{ptm} %não sei para que serve
%\rmdefault %selects a roman (i.e., serifed) font family
%\sfdefaultand %selects a sans serif font family
%\ttdefault %selects a monospaced (“typewriter”) font family 

%\rmfamily %selects a roman (i.e., serifed) font family
%\sffamily %selects a sans serif font family
%\ttfamily %selects a monospaced (“typewriter”) font family 
%Within each font family, the following declarations select the “series” (i.e., dark-ness or stroke width),
%\mdseries %regular
%\bfseries %bold
%and the “shape” (i.e., the form of the letters)
%\upshapeupright
%\slshapeslanted
%\itshapeitalic
%\scshapeCapsandSmallCap

%%% Configuração fina
%%% http://tug.ctan.org/tex-archive/macros/latex/contrib/microtype/microtype.pdf
%%\microtypecontext{
%%    activate={true,nocompatibility},%%% activate={true,nocompatibility} - activate protrusion and expansion
%%    final, %%% final - enable microtype; use "draft" to disable
%%    tracking=true,%%% tracking=true, kerning=true, spacing=true - activate these techniques
%%    protrusion=true,
%%    expansion,
%%    kerning=true,
%%    spacing=true,
%%    factor=1100,%%% factor=1100 - add 10% to the protrusion amount (default is 1000)
%%    stretch=10, %%% stretch=10, shrink=10 - reduce stretchability/shrinkability (default is 20/20)
%%    shrink=10,
%%    spacing=nonfrench
%%    }
%%

%%%----------------------------------------------------------------------------------------
%%%	Especificações do Sumário 
%%%----------------------------------------------------------------------------------------
\setcounter{tocdepth}{1} % Profundidade do sumário
\renewcommand{\contentsname}{SUMÁRIO}


%%%----------------------------------------------------------------------------------------
%%%	PARAGRAPH SPACING SPECIFICATIONS
%%%----------------------------------------------------------------------------------------
\setlength{\parindent}{15mm} % Retira indentação dos parágrafos
\setlength{\parskip}{2.5mm} % Whitespace between paragraphs
\usepackage{indentfirst}


%%%----------------------------------------------------------------------------------------
%%%	SECTION TITLE SPECIFICATIONS
%%%----------------------------------------------------------------------------------------

\usepackage{titlesec} % Required for modifying section titles

%%%----------------------------------------------------------------------------------------
%%%	REFERÊNCIA
%%%----------------------------------------------------------------------------------------




%%% Número e nome dos capítulos na mesma linha
\titleformat{\chapter} [hang]
{\sffamily\large\bfseries} % Title font customizations
{\chaptertitlename\ \thechapter.} % chapter number
{5pt} % Whitespace between the number and title
{\large} % Title font size
\titlespacing*{\chapter}{0mm}{5mm}{0mm} % Left, top and bottom spacing around the title

\titleformat{\section} %[block] % Customize the \section{} section title
{\sffamily\large\bfseries} % Title font customizations
{} % Section number
{0pt} % Whitespace between the number and title
{\large} % Title font size
\titlespacing*{\section}{0mm}{5mm}{0mm} % Left, top and bottom spacing around the title

\titleformat{\subsection} % Customize the \subsection{} section title
{\sffamily\large\bfseries} % Title font customizations
{\thesubsection} % Subsection number
{5pt} % Whitespace between the number and title
{\large} % Title font size
\titlespacing*{\subsection}{0mm}{5mm}{0mm} % Left, top and bottom spacing around the title

\titleformat{\subsubsection} % Customize the \subsection{} section title
{\sffamily\large\bfseries} % Title font customizations
{\thesubsubsection} % Subsection number
{5pt} % Whitespace between the number and title
{\large} % Title font size
\titlespacing*{\subsubsection}{0mm}{5mm}{0mm} % Left, top and bottom spacing around the title

\renewcommand{\chaptername}{CLÁUSULA} % Altera nomes dos capítulos de "Chapter" para "clausula"
%%%\renewcommand{\thechapter}{\Roman{chapter}} %Altera numeração de capítulos para números romanos

\def\thesection{\arabic{section}}

%%%% 2 capitulos na mesma pagina
\usepackage{etoolbox}
\makeatletter
\patchcmd{\chapter}{\if@openright\cleardoublepage\else\clearpage\fi}{}{}{}
\makeatother


%%%----------------------------------------------------------------------------------------
%%%	Especificações das listas e numerações 
%%%----------------------------------------------------------------------------------------
%\setcounter{secnumdepth}{9} % Profundidade da numeração


%\renewcommand*{\theenumi}{\thesection.\arabic{enumi}} % Permite numeração continuada a partir dos capítulos e seções.
%\renewcommand*{\theenumii}{\theenumi.\arabic{enumii}} % Permite numeração continuada a partir dos capítulos e seções.
%\usepackage{remreset} % OBSOLETE Para numeração de seções continuamente, independente dos capítulos

%\makeatletter 
%  \@removefromreset{section}{chapter}
%\makeatother


%%%-------COMENTARIOS---------------------------------------------------------------------------------
\iffalse
    teste
\fi


%%%-----------------------------------------------------------
%%%	Cabeçalho e Rodapé
%%%-----------------------------------------------------------
%% Redefine the plain page style
\fancypagestyle{plain}{%
    \fancyhf{}
    \renewcommand{\headrulewidth}{0.4pt}
    \fancyfoot[L]{}
    \fancyfoot[C]{}
    \fancyhead[R]{\sffamily{PROJETO PDI: \ApelidoProjeto\ -\ Contrato nº \NumeroContrato}}
    \renewcommand{\footrulewidth}{0.4pt}
    \pagenumbering{arabic}
    \fancyfoot[L]{}
    \fancyfoot[C]{\fontsize{9}{9} \sffamily{Chancelas e Rubricas:}}
    \fancyfoot[R]{\fontsize{9}{9} \sffamily {Página\ \thepage\ de }}

}

%% Redefine the empty  page style
\fancypagestyle{empty}{%
    \fancyhf{}
    \renewcommand{\headrulewidth}{0.4pt}
    \fancyfoot[C]{}
    \fancyhead[R]{}
    \renewcommand{\footrulewidth}{0.4pt}
    \pagenumbering{arabic}
    \fancyfoot[L]{\fontsize{9}{9} \sffamily{Chancelas e Rubricas:}}
    \fancyfoot[C]{}
    \fancyfoot[R]{\fontsize{9}{9} \sffamily{Página\ \thepage\ de }}
}

%%%----------------------------------------------------------------------------------------
%%%	Identação
%%%----------------------------------------------------------------------------------------
\setlength\parindent{0pt} % Remove identação de todos os parágrafos

\newcommand*\wildcard[2][7.5cm]{\vspace*{1.5cm}\parbox{#1}{\hrulefill\par#2}}


%%%----------------------------------------------------------------------------------------
%%%	DEFINICAO DE CORES 
%%%----------------------------------------------------------------------------------------
\tcbuselibrary{breakable, skins}
\definecolor{mygray}{gray}{0.8}
\definecolor{myyellow}{RGB}{247,163,15}
\definecolor{myorange}{RGB}{239,107,1}
\definecolor{mygreen}{RGB}{37,152,33}

\DeclareRobustCommand{\hlgray}[1]{{\sethlcolor{mygray}\hl{#1}}}
\DeclareRobustCommand{\hlyellow}[1]{{\sethlcolor{myyellow}\hl{#1}}}
\DeclareRobustCommand{\hlorange}[1]{{\sethlcolor{myorange}\hl{#1}}}
\DeclareRobustCommand{\hlgreen}[1]{{\sethlcolor{mygreen}\hl{#1}}}


\renewcommand{\baselinestretch}{1.3} % Configura distância horizontal entre as linhas

\newtcolorbox{lmarginbox}{blanker, breakable, right=2cm, borderline west={1pt}{0pt}{black}}