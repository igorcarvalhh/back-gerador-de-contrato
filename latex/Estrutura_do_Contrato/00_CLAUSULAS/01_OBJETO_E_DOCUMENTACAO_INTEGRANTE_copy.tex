\causula{CLÁUSULA PRIMEIRA - OBJETO E DOCUMENTAÇÃO INTEGRANTE}

\xx O presente \textbf{\textbf{CONTRATO}} tem por objeto a execução, pela(s) \textbf{\textbf{EXECUTORA(S)}}, de projeto de Pesquisa, Desenvolvimento e Inovação (“\textbf{PDI}”), identificado, em sua fase de Desenvolvimento Experimental ("DE"), e TRL (Nível de Prontidão Tecnológica) 5, pelo código ANEEL \textbf{\CodAneelProjeto}, e intitulado “\TituloCompletoProjeto”.

\xxx O referido projeto de pesquisa é dividido em três fases da cadeia de inovação:

1) Desenvolvimento experimental ("\textbf{DE – TRL 5}”), registrado com o código \textbf{\CodAneelProjeto};

2) Cabeça de série e lote pioneiro ("\textbf{CS/LP – TRL 8}”), registrado com o código ANEEL \textbf{\CodAneelProjeto};

3) Inserção no Mercado ("\textbf{IM – TRL 9}”), registrado com o código ANEEL \textbf{\CodAneelProjeto};

\xxx O referido projeto de pesquisa tem por objetivo \ObjetivoDoProjeto. Serão entregues pela(s) \textbf{\textbf{EXECUTORA(S)}} produtos conforme detalhamento exposto no \hyperlink{1.6}{item 1.6} deste contrato, e complementarmente, respeitando a Especificação Técnica (Anexo I) e Plano de Trabalho (Anexo II);

\xxx O referido projeto de pesquisa será adiante denominado simplesmente “\textbf{\textbf{PROJETO}}”.

\xx Nos termos deste \textbf{CONTRATO}, a(s) \textbf{EXECUTORA(S)} se obrigam a prestar todos os serviços e a fornecer todo o ferramental, documentação e materiais necessários à perfeita execução do objeto contratual, em conformidade com seus anexos e com todas as normas aplicáveis ao objeto deste \textbf{CONTRATO}, em especial o pleno atendimento a todos os módulos dos Procedimentos do Programa de Pesquisa e Desenvolvimento – PROP\&D aprovado pela Resolução Normativa nº 754/2016 e pela Lei 9.991 de 24 de julho de 2000, e de todos módulos dos Procedimentos do Programa de Pesquisa, Desenvolvimento e Inovação – PROPDI aprovado pela Resolução Normativa nº 1.045/2022, adiante denominados \textbf{PROPDI}, assim como quaisquer alterações posteriores.

\xx A(s) \textbf{EXECUTORA(S)} declaram que o objeto do \textbf{PROJETO} está enquadrado entre os que compõem o seu objeto social e que detêm conhecimento e experiência na execução dos mesmos, bem como que possui todos os registros e licenças necessários para sua realização, inexistindo qualquer restrição ou impedimento a respeito.

\xx São partes integrantes deste \textbf{CONTRATO}, além do seu texto principal:

\xxx O Anexo I, adiante denominado simplesmente “\textbf{ESPECIFICAÇÃO TÉCNICA}”;

\xxx O Anexo II adiante denominado simplesmente "\textbf{PLANO DE TRABALHO}";

\xxx O Anexo III adiante denominado simplesmente "\textbf{PLANO FINANCEIRO}";

\xxx O Anexo IV adiante denominado simplesmente "Conjunto de declarações de disponibilidade, aptidão e acordo de confidencialidade dos recursos humanos do projeto";

\xxx Anexo V adiante denominado “PL.SI.02.01 – Política do Sistema de Gestão Integrado TAESA (SGIT)”

\xxx Anexo VII adiante denominado “\textbf{Termo de Cessão de Direitos e Obrigações}”.

\xx Na hipótese de conflito entre as disposições constantes no corpo deste \textbf{CONTRATO} e quaisquer documentos anexos acima mencionados ou correlatos, tais como e sem se limitar a atas de reuniões, e-mails e outros, o disposto no corpo deste \textbf{CONTRATO} deverá prevalecer.

\xxx Os documentos anexos serão subordinados ao \textbf{CONTRATO} e prevalecerão, uns sobre os outros, de acordo com a ordem em que se apresentam neste \textbf{CONTRATO}.

\xxx Os demais documentos, tais como e sem se limitar a atas de reuniões, e-mails e outros, serão subordinados ao \textbf{CONTRATO} e a seus anexos e somente terão qualquer validade quando for unânime a aceitação do mesmo pelos \textbf{PARTÍCIPES}, prevalecendo uns sobre os outros e por ordem cronológica, valendo o mais recente sobre o mais antigo. Destaca-se que em nenhuma hipótese qualquer documento sem aceitação unânime poderá ser subordinado a este \textbf{CONTRATO}, seus anexos e demais documentos.

\xxx Em nenhuma hipótese qualquer documento poderá substituir ou contradizer este \textbf{CONTRATO} e seus anexos, exceto termos aditivos a este \textbf{CONTRATO} formalmente assinados pelos \textbf{PARTÍCIPES}.

\xx O objeto do presente instrumento será executado, pela(s) \textbf{EXECUTORA(S)}, nas instalações definidas pela \textbf{PROPONENTE}, em etapas e fases listadas no \textbf{PLANO DE TRABALHO} e gerando os seguintes produtos, nessa ordem:

% gera_entregaveis_por_etapa

\xxx \textbf{\hlgray{Fase da cadeia de inovação: Desenvolvimento Experimental. \\ 
 Nível de prontidão tecnológica: TRL - 5 \\ 
 Etapa: 1\\ 
 Duração: 06 (seis) meses 
}}

\xxxx 1 (um) relatório de busca de Anterioridade, confirmando o ineditismo de todos os produtos esperados do projeto, realizada nas seguintes bases: Anais de eventos do setor elétrico tais como e sem se limitar a: Citenel, SENDI, SNPTEE e outros; Bancos de publicações de produção científica tais como e sem se limitar a: SCielo, Engineering Village, Scopus, etc; Bancos de patentes e registros tais como e sem se limitar a: INPI e USPTO;

\xxxx Apresentação pública dos resultados da etapa/da fase;

\xxxx Inserir todos os entregáveis planejados para a etapa de forma objetiva e mensurável.

\xxxx Estudo de viabilidade inicial, baseado nos projetos análogos já existentes;

\xxx \textbf{\hlgray{Fase da cadeia de inovação: Desenvolvimento Experimental. \\ 
 Nível de prontidão tecnológica: TRL - 5 \\ 
 Etapa: 2\\ 
 Duração: 06 (seis) meses 
}}

\xxxx Apresentação pública dos resultados da etapa/da fase;

\xxxx Inserir todos os entregáveis planejados para a etapa de forma objetiva e mensurável

\xxx \textbf{\hlgray{Fase da cadeia de inovação: Desenvolvimento Experimental. \\ 
 Nível de prontidão tecnológica: TRL - 5 \\ 
 Etapa: 3\\ 
 Duração: 06 (seis) meses 
}}

\xxxx Apresentação pública dos resultados da etapa/da fase;

\xxxx Inserir todos os entregáveis planejados para a etapa de forma objetiva e mensurável

\xxx \textbf{\hlgray{Fase da cadeia de inovação: Desenvolvimento Experimental. \\ 
 Nível de prontidão tecnológica: TRL - 5 \\ 
 Etapa: 4\\ 
 Duração: 06 (seis) meses 
}}

\xxxx Inserir todos os entregáveis planejados para a etapa de forma objetiva e mensurável;

\xxxx Manual de operação e manutenção;

\xxxx Registro de patente e de software junto ao INPI e documentação relacionada;

\xxxx XX Dissertação de Mestrado;

\xxxx XX Tese de Doutorado;

\xxxx XX Código(s) fonte(s) auditados;

\xxxx XX artigo(s) científico(s) publicado(s) em periódico(s) classificado(s) como Qualis A1, A2 ou B1;

\xxxx Apresentação pública dos resultados da etapa/da fase;

\xxxx Vídeo de 10 minutos, realizado por empresa especializada, para divulgação dos resultados do Projeto;

\xxxx Estudo de Viabilidade Econômica, comprovando a viabilidade econômicofinanceira na execução da fase da cadeia de inovação, tomando como referência os custos para execução  dela, os custos de implantação dos resultados e os benefícios financeiros de sua aplicação;

\xxxx Relatório Final formato ANEEL (PDF e XML) da pesquisa da fase (XX) detalhando as atividades desenvolvidas e as conclusões do trabalho.

\xxx \textbf{\hlgray{Fase da cadeia de inovação: Cabeça de Série e Lote Pioneiro. \\ 
 Nível de prontidão tecnológica: TRL - 8 \\ 
 Etapa: 1\\ 
 Duração: 06 (seis) meses 
}}

\xxxx Apresentação pública dos resultados da etapa/da fase;

\xxxx Inserir todos os entregáveis planejados para a etapa de forma objetiva e mensurável;

\xxx \textbf{\hlgray{Fase da cadeia de inovação: Cabeça de Série e Lote Pioneiro. \\ 
 Nível de prontidão tecnológica: TRL - 8 \\ 
 Etapa: 2\\ 
 Duração: 06 (seis) meses 
}}

\xxxx Inserir todos os entregáveis planejados para a etapa de forma objetiva e mensurável;

\xxxx Manual de operação e manutenção;

\xxxx Registro de patente e de software junto ao INPI e documentação relacionada;

\xxxx XX Dissertação de Mestrado;

\xxxx XX Tese de Doutorado;

\xxxx XX Código(s) fonte(s) auditados;

\xxxx XX artigo(s) científico(s) publicado(s) em periódico(s) classificado(s) como Qualis A1, A2 ou B1;

\xxxx Apresentação pública dos resultados da etapa/da fase

\xxxx Estudo de Viabilidade Econômica, comprovando a viabilidade econômicofinanceira na execução da fase da cadeia de inovação, tomando como referência os custos para execução dela, os custos de implantação dos resultados e os benefícios financeiros de sua aplicação;

\xxxx Relatório Final formato ANEEL (PDF e XML) da pesquisa da fase (XX) detalhando as atividades desenvolvidas e as conclusões do trabalho

\xxx \textbf{\hlgray{Fase da cadeia de inovação: Inserção em Mercado. \\ 
 Nível de prontidão tecnológica: TRL - 9 \\ 
 Etapa: 1\\ 
 Duração: 06 (seis) meses 
}}

\xxxx Apresentação pública dos resultados da etapa/da fase;

\xxxx Estudo de Viabilidade Econômica, comprovando a viabilidade econômicofinanceira na execução da fase da cadeia de inovação, tomando como referência os custos para execução dela, os custos de implantação dos resultados e os benefícios financeiros de sua aplicação;

\xxxx Estudo de Viabilidade Econômica, comprovando a viabilidade econômicofinanceira na execução do PROJETO, tomando como referência os custos para execução dele, os custos de implantação dos resultados e os benefícios financeiros de sua aplicação;

\xxxx Material publicitário e respectivos arquivos editáveis, tais como e sem se limitar a: folders, banners, canvas, website etc., contendo todos os resultados do PROJETO necessário para divulgação e inserção no mercado;

\xxxx 5 (cinco) apresentações públicas dos resultados, objetivando transferência do conhecimento à equipe da PROPONENTE, COOPERADA(S) e ao mercado, incluindo materiais de divulgação em mídia

% fim gera_entregaveis_por_etapa

\xx As atividades relacionadas ao \textbf{CONTRATO} serão necessariamente atividades de natureza criativa ou empreendedora, com fundamentação técnico-científica e destinadas à geração de conhecimento ou à aplicação inovadora de conhecimento existente, inclusive para investigação de novas aplicações.

\xx O escopo do presente \textbf{CONTRATO} poderá ser alterado a qualquer tempo, mediante celebração de termo aditivo, desde que não seja alterada a linha de pesquisa, de forma a não descaracterizar o \textbf{PROJETO} e mediante negociação entre os \textbf{PARTÍCIPES}    .

\xx A(s) \textbf{EXECUTORA(S)} responderão sempre solidariamente, sem distinção de ordem, por todos os atos praticados na execução do \textbf{PROJETO}, em especial pelas obrigações dele decorrentes, incluindo, contudo sem a elas se limitar, as penalidades e obrigações estabelecidas neste instrumento e as obrigações de ordem cível, fiscal, administrativa, trabalhista, previdenciária e ambiental, independentemente de a materialização ocorrer durante ou após a execução do \textbf{PROJETO}.

## O objeto deste \textbf{CONTRATO} deve estar posicionado na cadeia de inovação nas fases de Desenvolvimento Experimental (DE – TRL 5), Cabeça de Série (CS – TRL 7), Lote Pioneiro (LP – TRL 8) e/ou Inserção no Mercado (IM – TRL 9), sendo imperativa a entrega pela(s) \textbf{EXECUTORA(S)}, ao final do \textbf{PROJETO}, dos produtos desenvolvidos em estágio finalizado, com plena capacidade para inserção no mercado, sendo desde já, não acatada qualquer justificativa de imprevisibilidade inerente à pesquisa básica e/ou dirigida.
