\causula{CLÁUSULA DÉCIMA SEGUNDA – DA SUBCONTRATAÇÃO}

\xx A \textbf{EXECUTORA} não poderá subcontratar o escopo do presente \textbf{CONTRATO} sem a prévia anuência por escrito da \textbf{PROPONENTE} e \textbf{COOPERADA(S)};

\xx Ao pleitear subcontratação, a \textbf{EXECUTORA} deverá apresentar à \textbf{PROPONENTE} e \textbf{COOPERADA(S)} os documentos necessários ao exame da situação jurídica, econômica e técnico-profissional do pretendido Subcontratado, além de declaração deste reportando que conhece, aceita e se obriga a cumprir e respeitar todas as disposições deste \textbf{CONTRATO};

\xx A \textbf{EXECUTORA} deverá negociar os respectivos contratos a serem celebrados por ela e os Subcontratados, e deverá garantir que tais contratos conterão (a) os termos e condições do presente \textbf{CONTRATO} em tudo aquilo que não for conflitante com o escopo específico de cada um dos Subcontratados, incluindo sem limitação de requisitos do Escopo, regras de pagamento, requisitos de segurança e medicina do trabalho, exigências ambientais e (b) todos os requisitos e condições necessários para que a \textbf{PROPONENTE} e \textbf{COOPERADA(S)} obtenham financiamento;

\xx Os referidos documentos devem prever que quaisquer disputas ou controvérsias relacionadas ao cumprimento de obrigações, ou a qualquer outra previsão contratual deverão ser resolvidas diretamente entre a \textbf{EXECUTORA} e os Subcontratados, sem que a \textbf{PROPONENTE} e \textbf{COOPERADA(S)} incorram no pagamento de qualquer indenização, custo ou despesa nesse sentido;

\xx Os trabalhos executados pelos Subcontratados estarão sujeitos à fiscalização pela \textbf{PROPONENTE} e \textbf{COOPERADA(S)} assim como pela \textbf{EXECUTORA}, sem que isto acarrete qualquer responsabilidade à \textbf{PROPONENTE} e \textbf{COOPERADA(S)}, permanecendo, assim, a \textbf{EXECUTORA} como responsável pela referida fiscalização;

\xx A \textbf{EXECUTORA} será totalmente responsável, perante a \textbf{PROPONENTE} e \textbf{COOPERADA(S)}, por qualquer ato, omissão, responsabilidade ou Falha de qualquer dos Subcontratados, seu Pessoal ou terceiros de sua responsabilidade. Assim sendo, qualquer ato ou omissão realizado por qualquer dos Subcontratados, seu Pessoal ou terceiros de sua responsabilidade será considerado como praticado pela EXECUTORA para os fins deste \textbf{CONTRATO} e, consequentemente, como um inadimplemento da \textbf{EXECUTORA}, devendo ser imediatamente sanado pela \textbf{EXECUTORA};

\xx A \textbf{EXECUTORA} deverá assegurar à \textbf{PROPONENTE} e \textbf{COOPERADA(S)} todos os direitos que lhes são assegurados neste \textbf{CONTRATO}, não obstante eventuais condições menos vantajosas acordadas com os Subcontratados; e
Todas e quaisquer menções às responsabilidades e/ou às obrigações da \textbf{EXECUTORA}, no âmbito deste \textbf{CONTRATO}, serão também entendidas, conforme o caso, como obrigação de a \textbf{EXECUTORA} fazer com que os Subcontratados também se responsabilizem e/ou se obriguem.