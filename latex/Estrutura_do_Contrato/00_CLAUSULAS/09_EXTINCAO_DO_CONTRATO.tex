\causula{CLÁUSULA NONA - EXTINÇÃO DO \textbf{CONTRATO}}


\xx Cada um dos \textbf{PARTÍCIPES} terá, a seu exclusivo critério, o direito a resolver o presente \textbf{CONTRATO}, mediante o simples envio de notificação por escrito a outro \textbf{PARTÍCIPE}, quando da ocorrência de quaisquer das eventualidades listadas abaixo:

\begin{enumerate}[label=\alph*),]
    \item Quando o outro \textbf{PARTÍCIPE} tiver sua falência decretada, ou seja, objeto de processo de Recuperação Judicial ou Extrajudicial;
    \item Quando o outro \textbf{PARTÍCIPE} transferir a totalidade ou parte substancial de seu patrimônio ou bens ou parar de exercer sua atividade comercial;
    \item Por motivo de Força Maior ou Caso Fortuito, conforme mencionado no Artigo 393 do Código Civil brasileiro, entendido como todo e qualquer ato do qual resulte impossibilidade de cumprimento das disposições previstas neste \textbf{CONTRATO}, para o qual não tenha contribuído, voluntária ou involuntariamente, qualquer um dos \textbf{PARTÍCIPES} deste \textbf{CONTRATO};
          \begin{enumerate}
              \item[c.1)] Para fins de cumprimento deste dispositivo, a(s) \textbf{EXECUTORA(S)} deverão encaminhar à \textbf{COOPERADA(S)} e/ou a \textbf{PROPONENTE}, em até 2 (dois) dias úteis contados a partir da ocorrência do evento a ser caracterizado como força maior, relatório contendo a descrição do evento danoso ocorrido, bem como todas as medidas já tomadas e/ou a serem tomadas pela(s) \textbf{EXECUTORA(S)} para que o impacto de tal evento na execução do objeto contratual seja o menor possível.

          \end{enumerate}
    \item A qualquer tempo mediante acordo entre os \textbf{PARTÍCIPES}.

\end{enumerate}

\xx A \textbf{COOPERADA(S)} e a \textbf{PROPONENTE}, complementarmente, terá, a seu exclusivo critério e sem prejuízo da aplicação das penalidades e da apuração de eventuais prejuízos por perdas e danos decorrentes, o direito de resolver imediatamente o presente \textbf{CONTRATO}, mediante o simples envio de notificação por escrito à(s) \textbf{EXECUTORA(S)}, quando da ocorrência de quaisquer dos itens abaixo listados:

\begin{enumerate}[label=\alph*)]
    \item Quando (i) o atraso no cumprimento de quaisquer dos prazos estabelecidos, ou (ii) o descumprimento de qualquer das obrigações relativas ao presente \textbf{CONTRATO} e seus anexos, pela(s) \textbf{EXECUTORA(S)}, possibilitarem à \textbf{COOPERADA(S)} e/ou a \textbf{PROPONENTE} a aplicação de penalidades no montante de 1/3 (um terço) do valor total deste \textbf{CONTRATO}, conforme definido na CLÁUSULA QUINTA acima;
    \item Quando a(s) \textbf{EXECUTORA(S)} violarem dispositivo deste \textbf{CONTRATO} e deixarem de retificar tal violação no prazo de 30 (trinta) dias após a respectiva violação, independente da aplicação ou não das penalidades previstas na CLÁUSULA OITAVA acima;
    \item Caso haja alteração do controle acionário ou da estrutura da(s) empresa(s) \textbf{EXECUTORA(S)}, que interfira ou venha a interferir, direta ou indiretamente, no regular cumprimento do presente \textbf{CONTRATO} e seus anexos, o que será avaliado exclusiva e isoladamente pela \textbf{COOPERADA(S)} e a \textbf{PROPONENTE};
    \item Quando a(s) \textbf{EXECUTORA(S)} deixarem de apresentar à \textbf{COOPERADA(S)} e a \textbf{PROPONENTE} qualquer um dos produtos relacionados no \textbf{PLANO DE TRABALHO}, com atraso superior a 60 (sessenta) dias do prazo marco para entrega, e sem que haja uma justificativa prévia e acordo com a \textbf{EXECUTORA}.
    \item Caso ocorra qualquer violação, por parte da(s) \textbf{EXECUTORA(S)}, das disposições contidas no Código de Ética da \textbf{COOPERADA(S)}, da \textbf{PROPONENTE} e/ou nas Leis Anticorrupção;
    \item Incapacidade técnica, negligência, imprudência ou imperícia por parte da(s) \textbf{EXECUTORA(S)};
    \item Ao final de cada uma das etapas ou fases da cadeia de inovação.

\end{enumerate}

\xx A \textbf{COOPERADA(S)} e a \textbf{PROPONENTE} terão, ainda, o direito de resilir o presente \textbf{CONTRATO}, a seu exclusivo critério, por meio do envio de comunicação com antecedência mínima de 30 (trinta) dias corridos à(s) \textbf{EXECUTORA(S)}.

\xx Extinto o \textbf{CONTRATO}, a(s) \textbf{EXECUTORA(S)} têm um prazo de 10 (dez) dias corridos, a contar da extinção, para se retirar do local da prestação dos serviços e deixá-lo inteiramente desimpedido, quando este local for de propriedade ou uso da \textbf{COOPERADA(S)} e/ou da \textbf{PROPONENTE} bem como devolver à \textbf{COOPERADA(S)} e/ou a \textbf{PROPONENTE} todos documentos de propriedade da mesma, além de, ainda dentro do mesmo prazo, prestar contas de todo e qualquer obrigação pendente, inclusive com entrega da documentação prevista na SUBCLÁUSULA 3.28, deste \textbf{CONTRATO}.

\xx A extinção deste \textbf{CONTRATO} não afetará quaisquer direitos ou obrigações dos \textbf{PARTÍCIPES} ora pactuadas que possam ter se originado por força deste \textbf{CONTRATO}, anteriormente à rescisão ora referida.

\xx Nenhuma indenização, compensação ou ressarcimento por eventuais perdas e danos decorrentes será devida a qualquer dos \textbf{PARTÍCIPES}, caso a rescisão contratual tenha sido originada nos termos do item “c)” da SUBCLÁUSULA 9.1, nos termos SUBCLÁUSULA 9.3, ou por mútuo acordo entre os \textbf{PARTÍCIPES}.

\xx O presente contrato poderá ser extinto, no caso de determinação do agente regulador de Energia Elétrica (ANEEL) ou por determinação legal de destino de recurso de Pesquisa e Desenvolvimento a outro fim, desde que expressamente determinado pelo regulador ou agente federal. Nesta hipótese nenhuma indenização, compensação ou ressarcimento por eventuais perdas e danos decorrentes será devida a qualquer das PARTES.

\xx O presente contrato poderá ser extinto a qualquer momento, sem nenhuma perda das partes, no caso do relatório de busca de anterioridade do item 3.37 deste contrato indicar que o projeto não atende ao critério de originalidade exigido pela agência reguladora (ANEEL).