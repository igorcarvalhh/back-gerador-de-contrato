\causula{CLÁUSULA SEGUNDA - DA PROPRIEDADE INDUSTRIAL E UTILIZAÇÃO DOS RESULTADOS}

\xx Todos e quaisquer direitos econômicos e de propriedade intelectual com relação aos manuais, relatórios, dados técnicos, conclusões ou demais produtos advindos do desenvolvimento do \textbf{PROJETO} serão partilhados entre a \textbf{PROPONENTE}, \textbf{COOPERADA(S)} e a(s) \textbf{EXECUTORA(S)} proporcionalmente ao aporte financeiro no \textbf{PROJETO} objeto deste \textbf{CONTRATO} conforme exigido pelo \textbf{PROPDI} e descritos no artigo 9º, parágrafo 3º, da Lei nº 10.973 de 2004.

\xxx O valor do aporte financeiro de cada um dos \textbf{PARTICIPES} será definido no relatório final de auditoria contábil e financeira do projeto, que deverá conter, conforme determina o \textbf{PROPDI} e o Manual de Procedimentos Previamente Acordados para Auditoria Contábil e Financeira de Projetos, Planos e Programas de Pesquisa e Desenvolvimento – P\&D e Eficiência Energética – EE, as informações referentes aos gastos dos \textbf{PARTÍCIPES} que aportaram recursos na execução do \textbf{PROJETO}.

\xxxx A Auditoria deverá ser realizada por pessoa jurídica independente, inscrita na Comissão de Valores Mobiliários – CVM.

\xxx A(s) \textbf{EXECUTORA(S)} não exercerão o direito de cobrar quaisquer tipos de taxas e/ou royalties da \textbf{COOPERADA(S)} e da \textbf{PROPONENTE} ou de quaisquer sociedades relacionadas à \textbf{COOPERADA(S)} e à \textbf{PROPONENTE} pela utilização dos documentos, materiais, produtos ou resultados oriundos do desenvolvimento do \textbf{PROJETO} objeto deste \textbf{CONTRATO}.

\xxx Caso haja participação de instituição de pesquisa pública, essa tem direito à licença sem ônus e não exclusiva dos resultados da pesquisa para que os utilizem em pesquisas ou para fins didáticos.

\xx Os \textbf{PARTÍCIPES} reterão seus direitos de propriedade intelectual/industrial dos produtos desenvolvidos anteriormente ao \textbf{PROJETO}, que serão utilizados ou modificados durante a prestação dos serviços.

\xx A(s) \textbf{EXECUTORA(S)} somente poderão industrializar e comercializar os materiais ou equipamentos desenvolvidos sob a égide deste \textbf{CONTRATO} mediante celebração de \textbf{CONTRATO} específico, que estabelecerá os montantes referentes aos royalties a serem pagos à \textbf{COOPERADA(S)} e a \textbf{PROPONENTE}.

% Colocar espaço horizontal nesse parágrafo
\leftskip 30pt 
\textbf{PARÁGRAFO ÚNICO}: A exploração das tecnologias e produtos desenvolvidos no âmbito deste \textbf{CONTRATO} por terceiros, através de licença de exploração dos direitos, em caráter não exclusivo, poderão ser concedidas sob determinação da \textbf{COOPERADA(S)} e/ou \textbf{PROPONENTE} mediante celebração de \textbf{CONTRATO} específico. Também serão definidas no citado \textbf{CONTRATO} específico, as remunerações (royalties) a serem cobradas por essa exploração, as formas de auditoria e as reconstituições adquiridas de terceiros.

\xx A \textbf{COOPERADA(S)} e a \textbf{PROPONENTE} decidirão pela viabilidade e o interesse de se depositarem, no Brasil e no exterior, pedidos de patente sobre invenções, modelos de utilidade, marcas e/ou direitos autorais que resultarem da execução do \textbf{PROJETO} objeto deste \textbf{CONTRATO}, cabendo à \textbf{COOPERADA(S)} e à \textbf{PROPONENTE} a responsabilidade pelo registro da propriedade intelectual junto ao INPI e dos respectivos repasses financeiros.

\xx O direito autoral de Propriedade Intelectual de todos os softwares e tecnologias desenvolvidas e demais serviços entregáveis, a título universal e irretratável, são exclusivos da \textbf{COOPERADA(S)} e \textbf{PROPONENTE} no Brasil e em qualquer país, consoante Lei Nº 9.609/98 e 9.610/98, devendo as \textbf{EXECUTORA(S)} entregar a \textbf{COOPERADA(S)} e \textbf{PROPONENTE} os códigos fontes e a sua a documentação completa, em especial do código-fonte comentado, memorial descritivo, especificações funcionais internas, diagramas, fluxogramas e outros dados técnicos necessários à absorção da tecnologia. As \textbf{EXECUTORA(S)} assume(m) a total responsabilidade pela originalidade dos softwares a serem criados e cedidos, especialmente abstendo-se de compor os programas com componentes não licenciados, devendo-o indenizar a \textbf{COOPERADA(S)} e \textbf{PROPONENTE} por eventuais perdas e danos no caso desta falta.